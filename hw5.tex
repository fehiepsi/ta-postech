
% Default to the notebook output style

    


% Inherit from the specified cell style.




    
\documentclass{article}

    
    
    \usepackage{graphicx} % Used to insert images
    \usepackage{adjustbox} % Used to constrain images to a maximum size 
    \usepackage{color} % Allow colors to be defined
    \usepackage{enumerate} % Needed for markdown enumerations to work
    \usepackage{geometry} % Used to adjust the document margins
    \usepackage{amsmath} % Equations
    \usepackage{amssymb} % Equations
    \usepackage[mathletters]{ucs} % Extended unicode (utf-8) support
    \usepackage[utf8x]{inputenc} % Allow utf-8 characters in the tex document
    \usepackage{fancyvrb} % verbatim replacement that allows latex
    \usepackage{grffile} % extends the file name processing of package graphics 
                         % to support a larger range 
    % The hyperref package gives us a pdf with properly built
    % internal navigation ('pdf bookmarks' for the table of contents,
    % internal cross-reference links, web links for URLs, etc.)
    \usepackage{hyperref}
    \usepackage{longtable} % longtable support required by pandoc >1.10
    

    

    
    % Prevent overflowing lines due to hard-to-break entities
    \sloppy 
    % Setup hyperref package
    \hypersetup{
      breaklinks=true,  % so long urls are correctly broken across lines
      colorlinks=true,
      urlcolor=blue,
      linkcolor=darkorange,
      citecolor=darkgreen,
      }
    % Slightly bigger margins than the latex defaults
    
    \geometry{verbose,tmargin=1in,bmargin=1in,lmargin=1in,rmargin=1in}
    
    

    \begin{document}
    
    
    
    
    

    
    \subsection*{Analysis I - homework - week
5}\label{analysis-i---homework---week-5}
\addcontentsline{toc}{subsection}{Analysis I - homework - week 5}

    $\newcommand{\cl}{\operatorname{cl}}$ \textbf{1.} $(a)$ Show that a
Lipschitz map is uniformly continuous.

$(b)$ Find a bounded continuous function $f:\mathbb{R}\to \mathbb{R}$
which is not uniformly continuous.

$(c)$ Is the sum (product) of two Lipschitz continuous functions again
Lipschitz continuous?

$(d)$ Is the sum (product) of two uniformly continuous functions agian
uniformly continuous?

\textbf{Proof.} $(a)$ Suppose $f:A\subset \mathbb{R}^n \to \mathbb{R}^m$
is a Lipschitz map with constant $L$. If $L = 0$, then
$\|f(x) - f(y)\| = 0$ for all $x,y \in A$. So $f$ is a constant
function, hence is uniformly continuous. Now, suppose $L > 0$. For
$\epsilon > 0$, we put $\delta = \epsilon /L$. Then for all
$\epsilon > 0$, if $\|x - y\| < \delta$, then
$\|f(x) -f(y)\|\le L \|x-y\| < L\epsilon/L = \epsilon$. So $f$ is
uniformly continuous.

$(b)$ See Exercise 7.

$(c)$ This question is already answered in HW\#4. For the sum, yes. For
the product, no.

$(d)$ $(sum)$ Yes. We may use $\epsilon-\delta$ argument to solve this
problem. However, here we will use Exercise 4.a. Suppose
$f,g:A\subset \mathbb{R}^n \to \mathbb{R}^m$ are two uniformly
continuous functions. Let $\{x_n\}$, $\{y_n\}$ be two sequences in $A$
such that $(x_n-y_n) \to 0$ as $n\to \infty$. By $f,g$ are uniformly
continuous, we have $(f(x_n) -f(y_n)) \to 0$ and $(g(x_n)-g(y_n)) \to 0$
as $n\to \infty$. So $((f+g)(x_n) - (f+g)(y_n)) \to 0$ as $n\to\infty$.
This shows that $f+g$ is also uniformly continuous.

$(product)$ No. For example, let $f,g:\mathbb{R}\to \mathbb{R}$ defined
by $f(x) = x$, $g(x) = \sin(x)$. Clearly $f$ is uniformly continuous.
Similar to Exercise 6.b, we have the function $g(x)=\sin(x)$ is also
uniformly continuous. But their product $(fg)(x) = x\sin(x)$ is not
uniformly continuous (see Exercise 6.d). $\Box$

    \textbf{2.} Given that temperature on the surface of the earth is a
continuous function, prove that on any great circle of the earth there
are two antipodal points with the same temperature.

\textbf{Proof.} Let $T$ be the temperature function restricted to a
given great circle. We may consider $T$ as a continuous function from
$[0,2\pi]$ to $\mathbb{R}$ with $T(0) = T(2\pi)$. Let
$f:[\pi ,2\pi]\to\mathbb{R}$ defined by $f(x) = T(x) - T(x-\pi)$. Then
we have $f$ is continuous, $f(\pi) = T(\pi) - T(0)$, and
$f(2\pi) = T(2\pi) -T(\pi) = T(0) - T(\pi) = -f(\pi)$. Without loss of
generation, we may suppose that $f(\pi) \le 0$, hence $f(2\pi) \ge 0$.
Because $f$ is continuous and $[\pi, 2\pi]$ is connected, we have
$f([\pi,2\pi])$ is connected. So
$[f(\pi),f(2\pi)]\subset f([\pi,2\pi])$. In particular, by
$0 \in [f(\pi),f(2\pi)]$, there exists $a\in [\pi,2\pi]$ such that
$f(a) = 0$. This means that $T(a) = T(a-\pi)$. The pair $a$, $a-\pi$
corresponds to two antipodal points in this given circle, and they have
the same temperature. $\Box$

    \textbf{3.} Let $f:[0,1]\to [0,1]$ be a continuous function. Prove that
there exists a point $x\in [0,1]$ such that $f(x) = x$.

\textbf{Proof.} Let $A = \{x\in [0,1] \mid f(x) \ge x\}$ and let
$a = \sup A$. By definition of supremum, there is a sequence
$\{x_n\}\subset A$ converging to $a$. By the continuity of $f$, we have
$f(x_n)\to f(a)$ as $n\to \infty$. By $f(x_n) \ge x_n$ for all $n$, we
have $f(a) \ge a$.

If $a = 1$, then by $f(1) \le 1$, we get $f(1) = 1$ and $1$ is the point
we want. In the case $a < 1$, we note that $f(x) < x$ for all $x > a$.
Let $\{x_n\}\subset [0,1]$ be a decreasing sequence which converges to
$a$. By the same argument as above, we can show that $f(a) \le a$. In
conclusion, we have $f(a) = a$. $\Box$

    \textbf{4.} Let $f:A\subset \mathbb{R}^n\to \mathbb{R}^m$.

$(a)$ Prove $f$ is uniformly continuous on $A$ if and only if for every
pair of sequences $x_k$, $y_k$ of $A$ such that $(x_k-y_k) \to 0$, we
have $f(x_k)-f(y_k)\to 0$.

$(b)$ Let $f$ be uniformly continuous, and $x_k$ be a Cauchy sequence of
$A$. Show that $f(x_k)$ is a Cauchy sequence.

$(c)$ Let $f$ be uniformly continuous. Show $f$ has a unique extension
to a continuous function on $\bar{A} = \cl(A)$.

\textbf{Proof.} $(a)$ Suppose that $f$ is uniformly continuous and let
$\{x_k\}$, $\{y_k\}$ be two sequences in $A$ such that
$(x_k - y_k) \to 0$. Let $\epsilon > 0$. By the uniform continuity of
$f$, there exists $\delta > 0$ such that $\|f(x) - f(y)\| < \epsilon$ if
$|x-y|< \delta$. Let $N\in \mathbb{N}$ such that
$\|x_k - y_k\| < \delta$ for all $k > N$. Then we have
$\|f(x_k) -f(y_k)\| < \epsilon$ for all $k > N$. So
$(f(x_k) - f(y_k)) \to 0$ as $k\to \infty$.

For the reverse side, suppose on the contrary that $f$ is not uniformly
continuous on $A$. So there exists $\epsilon > 0$ such that for each
$\delta > 0$, there exists $x(\delta), y(\delta)$ such that
$\|x(\delta)- y(\delta)\|< \delta$ and
$\|f(x(\delta)) -f(y(\delta))\| \ge \epsilon$. For each $k$, let
$x_k = x(1/k)$ and $y_k = y(1/k)$, then we have $\|x_k - y_k\| < 1/k$,
which implies that $(x_k - y_k) \to 0$ as $k\to \infty$. By
$\|f(x_k) -f(y_k)\| \ge \epsilon$ for all $k$, it cannot happen that
$f(x_k)-f(y_k) \to 0$ as $k\to \infty$. We conclude.

$(b)$ Let $\epsilon > 0$. By the uniform continuity of $f$, there exists
$\delta > 0$ such that $\|f(x) - f(y)\| < \epsilon$ if
$\|x-y\| < \delta$. Because $\{x_k\}$ is a Cauchy sequence, there is
$N \in \mathbb{N}$ such that $\|x_m - x_n\| < \delta$ for all
$m,n \ge N$. So we have $\|f(x_m)-f(x_n)\| < \epsilon$ for all
$m,n \ge N$. This means that $\{f(x_k)\}$ is a Cauchy sequence.

$(c)$ Suppose that $g,h$ be two continuous function on $\bar{A}$ such
that $g(x) = h(x) = f(x)$ for all $x\in A$. Let $x\in \bar{A}$, there
exists a sequence $\{x_n\}\subset A$ such that $x_n\to x$ as
$n\to\infty$. In particular, we have $\{x_n\}$ is a Cauchy sequence in
$A$, so $\{f(x_n)\}$ is a Cauchy sequence in $\mathbb{R}^m$ (by $(b)$).
Due to the completeness of $\mathbb{R}^m$, $\{f(x_n)\}$ converges to a
point $y \in \mathbb{R}^m$. Note that $g(x_n) = f(x_n)$ for all $n$,
hence $g(x_n) \to y$ as $n\to \infty$. By the continuity of $g$, we have
$g(x_n)\to g(x)$ as $n\to \infty$. So $g(x) = y$. Similarly,
$h(x) = y = g(x)$. This is true for all $x\in \bar{A}$, so the
continuous extension of $f$ to $\bar{A}$ (if has) is unique.

For the existence, let $\bar{f}$ be the extension of $f$ on $\bar{A}$
defined by $\bar{f}(x) =\lim_{n\to \infty}f(x_n)$ with
$\{x_n\} \subset A$ and $x_n \to x$. If $\{y_n\} \subset A$ and
$y_n\to x$, then $\lim_{n\to\infty} (x_n -y_n) = 0$, hence
$\lim_{n\to\infty}(f(x_n)-f(y_n)) = 0$, hence
$\lim_{n\to \infty}f(x_n) = \lim_{n\to \infty}f(y_n)$. So $\bar{f}$ is
well-defined.

Now the remaining thing is to show that $\bar{f}$ is continuous on
$\bar{A}$. Suppose on the contrary that $\bar{f}$ is not continuous at
$x\in \bar{A}$. Then there exists $\epsilon > 0$ such that for all $n$,
there exists $x_n\in\bar{A}$ statisfying $\|x_n-x\| < 1/n$ and
$\|\bar{f}(x_n) - \bar{f}(x)\| \ge \epsilon$. The way we define
$\bar{f}$ suggests that for each $n$, there exists $y_n\in A$ such that
$\|y_n - x_n\| < 1/n$ and $\|f(y_n) - \bar{f}(x_n)\| < \epsilon/2$. So
we have $\|y_n - x\| < 2/n$ and $\|f(y_n)-\bar{f}(x)\| > \epsilon/2$. We
get a sequence $\{y_n\}\subset A$, $y_n \to x$, but $\{f(y_n)\}$ does
not converge to $\bar{f}(x)$. This is a contradiction. $\Box$

    \textbf{5.} If $A,B\subset \mathbb{R}^n$, define $A+B$ to be the set of
all sums $x+y$ with $x\in A$, $y\in B$. If $A$ and $B$ are compact,
prove that $A + B$ is compact.

\textbf{Proof.} Let $\{x_n+y_n\}$ be a sequence in $A+B$ with
$x_n \in A$, $y_n \in B$ for all $n$. By the compactness of $A$, there
exists a subsequence $\{x_{n_k}\}$ of $\{x_n\}$ which converges to
$a \in A$. By the compactness of $B$, there exists a subsequence
$\{y_{n_{k_t}}\}$ of $\{y_{n_k}\}$ which converges to $b\in B$. So we
have $x_{n_{k_t}}+ y_{n_{k_t}} \to a + b$ as $t\to \infty$, which means
that the sequence $\{x_n+y_n\}$ has a subsequence converges to
$a+b \in A +B$. In conclusion, we have $A+B$ compact. $\Box$

    \textbf{6.} Which of the following functions on $\mathbb{R}$ are
uniformly continuous?

$(a)$ $f(x) := \frac{1}{x^2 +1}$,

$(b)$ $f(x) := \cos(x)$,

$(c)$ $f(x) := \frac{x^2}{x^2+2}$,

$(d)$ $f(x) := x\sin(x)$.

\textbf{Proof.} $(a)$ This function is Lipschitz continuous, so it is
uniformly continuous. Indeed, we have \[\begin{aligned}
|f(x) - f(y)| &= \left|\frac{1}{x^2+1}- \frac{1}{y^2+1}\right| \\
& = |x-y|\frac{|x+y|}{(x^2+1)(y^2+1)} \\
&\le |x-y| \frac{|x+y|}{\sqrt{(x^2+1)(y^2+1)}} \\
&\le |x-y| \frac{\sqrt{2(x^2+y^2)}}{\sqrt{x^2 + y^2+1}} \\
& \le \sqrt{2}|x-y|.
\end{aligned}
\]

$(b)$ This function is Lipschitz continuous, so it is uniformly
continuous. Indeed, we have \[\begin{aligned}
|f(x) - f(y)| &= |\cos(x) - \cos(y)|\\
&= 2\left|\sin\left(\frac{x+y}{2}\right)\right|\left|\sin\left(\frac{x-y}{2}\right)\right| \\
&\le 2\left|\sin\left(\frac{x-y}{2}\right)\right| \\
&\le 2\frac{|x-y|}{2}\\
&= |x-y|.
\end{aligned}
\]

$(c)$ By $\frac{x^2}{x^2+2} = 1- \frac{2}{x^2+2}$, this function is
similar to the function in $(a)$. With a similar argument, we can show
that $f$ is Lipschitz, hence $f$ is uniformly continuous.

$(d)$ This function is not uniformly continuous. Indeed, for each
$k\in \mathbb{N}$, let $n_k$ large enough such that \[
\left(2n_k\pi + \frac{1}{k}\right) \sin\left(\frac{1}{k}\right) \ge 1,
\] and put $x_k = 2n_k\pi + \frac{1}{k}$ and $y_k = 2n_k\pi$. Then
$(x_k - y_k) \to 0$ as $k\to \infty$. However, for each $k$, we have
$f(y_k) = 0$ and
\[f(x_k) = x_k\sin(x_k) = \left(2n_k\pi + \frac{1}{k}\right) \sin\left(2n_k\pi + \frac{1}{k}\right) \ge 1.\]

So we don't have that $(f(x_k)- f(y_k)) \to 0$ as $k\to \infty$. By
Exercise 4.a, this function is not uniformly continuous. $\Box$

    \textbf{7.} Let $f:(0,1)\to \mathbb{R}$ be a continuous function. Prove
or disprove: If $f$ is bounded, then $f$ is uniformly continuous.

\textbf{Proof.} Disprove. For example, let $f$ defined by
$f(x) = \sin(1/x)$. Clearly $f$ is bounded. Suppose on the contrary that
$f$ is uniformly continuous. By Exercise 4, we have $\{f(x_n)\}$ is a
Cauchy sequence for all Cauchy sequence $\{x_n\}$ in $(0,1)$. Now, let
$x_n = 1/(\pi n + \pi/2)$, then $f(x_n) = 1$ if $n$ is even and
$f(x_n) = -1$ if $n$ is odd. So $\{f(x_n)\}$ cannot be a Cauchy
sequence. We get a contradiction. $\Box$

    \textbf{8.} Give an example that $f:\mathbb{R}^n \to \mathbb{R}^m$ is
continuous function and $V$ is a closed subset of $\mathbb{R}^n$, but
$f(V)$ is not closed subset of $\mathbb{R}^m$.

\textbf{Proof.} Let $f: \mathbb{R} \to \mathbb{R} $ defined by
$f(x) = 1$ if $x \ge 1$ and $f(x) = 1/x$ if $x > 1$. Let
$V = [1,\infty)$, which is a closed subset of $\mathbb{R}$. Here, we
have $f$ is continuous and $f(V) = (0,1]$, which is not closed in
$\mathbb{R}$. $\Box$

    \textbf{9.} If $f$ is a continuous mapping of a metric space $X$ into a
metric space $Y$, prove that \[ f(\cl(E)) \subset \cl(f(E))\] for every
set $E\subset X$. Show, by an example, that $f(\cl(E))$ can be a proper
subset of $\cl(f(E))$; that is, $f(\cl(E)) \ne \cl(f(E))$.

\textbf{Proof.} By $\cl(f(E))$ is a closed subset of $Y$, we have
$f^{-1}(\cl(f(E)))$ is closed in $X$. Clearly we have
\[E\subset f^{-1}(f(E))\subset f^{-1}(\cl(f(E))).\]

So $\cl(E) \subset f^{-1}(\cl(f(E)))$, which implies that
$f(\cl(E)) \subset \cl(f(E))$.

The inclusion can be proper by the following example: Let
$f:\mathbb{R}\to \mathbb{R}$ defined by $f(x) = 1$ if $x\ge 1$ and
$f(x) = 1/x$ if $x > 1$. Let $E = \mathbb{R}$, then
\[f(E)= f(\cl(E)) = f(\mathbb{R}) = (0,1]\] but $\cl(f(E)) = [0,1]$.
$\Box$

    \textbf{10.} If $E$ is a nonempty subset of a metric space $X$, define
the distance from $x\in X$ to $E$ by
\[ \rho_E(x) := \inf_{z\in E} d(x,z)\] where, $d(x,z)$ is a distance
from $x$ to $z$.

$(a)$ Prove that $\rho_E(x) = 0$ if and only if $x\in\cl (E)$.

$(b)$ Prove that $\rho_E$ is a uniformly continuous function on $X$.

\textbf{Proof.} $(a)$ Suppose that $\rho_E(x) = 0$. Then by definition
of infimum, there exists a sequence $\{z_n\} \subset E$ such that
$d(x,z_n) \to \rho_E(x) = 0$ as $n\to\infty$. So $z_n \to x$ as
$n\to \infty$, which means that $x \in \cl(E)$.

For the reverse side, suppose that $x\in \cl(E)$. There is a sequence
$\{z_n\}\subset E$ converging to $x$. So $d(x,z_n) \to 0$ as
$n \to \infty$. For all $\epsilon >0$, there exists $n$ such that
$d(x,z_n) < \epsilon$. So $\inf_{z\in E} d(x,z) = 0$, which means that
$\rho_E(x) = 0$.

$(b)$ For $x,y\in X$ and for all $z\in E$, we have
\[\rho_E(x) \le d(x,z) \le d(x,y) + d(y,z).\]

By $\rho_E(x) \le d(x,y) + d(y,z)$ is true for all $z\in E$, we have
\[\rho_E(x) \le d(x,y) + \rho_E(y).\]

In other words, \[\rho_E(x) - \rho_E(y) \le d(x,y).\]

Similarly, we have \[\rho_E(y) - \rho_E(x) \le d(y,x)=d(x,y).\]

So \[|\rho_E(x) - \rho_E(y)| \le d(x,y)\] for all $x,y\in X$.

Now, apply Exercise 4.a, if $\{x_n\}$, $\{y_n\}$ are two sequences in
$X$ such that $d(x_n,y_n) \to 0$, then by
\[ |\rho_E(x_n) - \rho_E(y_n)| \le d(x_n,y_n) \] for all $n$, we have
$\rho_E(x_n) - \rho_E(y_n) \to 0$ as $n\to \infty$. So $f$ is uniformly
continuous. $\Box$

\emph{Note.} In Exercise 4.a, we don't use any other properties of
$A\subset \mathbb{R}^n$ but it is a metric space (and replace $(x-y)$
and $\|x-y\|$ by $d(x,y)$). Here $\rho_E$ is a function from the metric
space $X$ to $\mathbb{R}$.


    % Add a bibliography block to the postdoc
    
    
    
    \end{document}
