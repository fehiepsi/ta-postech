
% Default to the notebook output style

    


% Inherit from the specified cell style.




    
\documentclass{article}

    
    
    \usepackage{graphicx} % Used to insert images
    \usepackage{adjustbox} % Used to constrain images to a maximum size 
    \usepackage{color} % Allow colors to be defined
    \usepackage{enumerate} % Needed for markdown enumerations to work
    \usepackage{geometry} % Used to adjust the document margins
    \usepackage{amsmath} % Equations
    \usepackage{amssymb} % Equations
    \usepackage[mathletters]{ucs} % Extended unicode (utf-8) support
    \usepackage[utf8x]{inputenc} % Allow utf-8 characters in the tex document
    \usepackage{fancyvrb} % verbatim replacement that allows latex
    \usepackage{grffile} % extends the file name processing of package graphics 
                         % to support a larger range 
    % The hyperref package gives us a pdf with properly built
    % internal navigation ('pdf bookmarks' for the table of contents,
    % internal cross-reference links, web links for URLs, etc.)
    \usepackage{hyperref}
    \usepackage{longtable} % longtable support required by pandoc >1.10
    \usepackage{booktabs}  % table support for pandoc > 1.12.2
    

    

    
    % Prevent overflowing lines due to hard-to-break entities
    \sloppy 
    % Setup hyperref package
    \hypersetup{
      breaklinks=true,  % so long urls are correctly broken across lines
      colorlinks=true,
      urlcolor=blue,
      linkcolor=darkorange,
      citecolor=darkgreen,
      }
    % Slightly bigger margins than the latex defaults
    
    \geometry{verbose,tmargin=1in,bmargin=1in,lmargin=1in,rmargin=1in}
    
    

    \begin{document}
    
    
    
    
    

    
    \subsection*{Analysis I - homework - week 14
(HW\#13)}\label{analysis-i---homework---week-14-hw13}
\addcontentsline{toc}{subsection}{Analysis I - homework - week 14
(HW\#13)}

    \textbf{1.} Suppose $F$ and $G$ are differentiable functions on $[a,b]$,
$f=F'$, and $g=G'$. Prove that if $f$ and $g$ are Riemann integrable,
then
\[\int_a^b F(x)g(x) \,dx = F(b)G(b) - F(a)G(a) - \int_a^b f(x)G(x)\,dx.\]

    \textbf{Proof.} \emph{Lemma.} If $T:[a,b]\to \mathbb{R}$ be
differentiable and $t\equiv T'$ is integrable. Then
\[\int_a^b t(x) \,dx = T(b) - T(a).\]

First, we prove this lemma. Let $P$ be a partition of $[a,b]$, which
means $P = \{x_0=a,x_1,\ldots,x_{n-1},x_n=b\}$, where $x_i < x_j$ if
$i < j$. By the mean value theorem, in each $i$ ($0\le i < n$), we have
$T(x_{i+1})-T(x_i) = (x_{i+1}-x_i)T'(c_i)$ for some
$c_i \in (x_i,x_{i+1})$. So
\[T(b) - T(a) = \sum_{i=1}^{n-1}(T(x_{i+1})-T(x_i) ) = \sum_{i=1}^{n-1} (x_{i+1}-x_i)t(c_i).\]

Hence we have $L(t,P) \le T(b) -T(a) \le U(t,P)$ for all $P$. By the
integrability of $t$, we conclude that
\[\int_a^b t(x) \,dx = T(b) - T(a).\]

Return to our problem, put $H=FG$, then $H$ is differentiable and
$H'=F'G+FG' = fG + Fg$. By $f$ is integrable, the set of discontinuous
point of $f$ has measure $0$, so the set of discontinuous points of $fG$
has measure zero. Similarly, the set of discontinuous points of $Fg$ has
measure zero. By the union of two measure zero sets has measure zero,
the set of discontinuous points of $H'$ has measure zero. So $H'$ is
integrable. By the above lemma, we have
\[\int_a^bH'(x)\,dx = H(b) - H(a).\]

In other words, we have
\[\int_a^b (F(x)g(x) + f(x)G(x))\,dx = F(b)G(b) - F(a)G(a),\] which is
exactly what we want to prove. $\Box$

    \textbf{2.} Let $f:[a,b]\to \mathbb{R}$ be continuous and differentible
on $(a,b)$. Assume $f(a) = 0$, $f(b) = -1$, and $\int_a^bf(x)\,dx = 0$.
Prove that there is a $c\in (a,b)$ such that $f'(c) = 0$.

    \textbf{Proof.} Put $F(t) = \int_a^t f(x)\,dx$ for $t\in [a,b]$. Then we
have $F(a) = F(b) = 0$. By the mean value theorem, we have $F'(d) = 0$
for some $d\in (a,b)$. So we have $f(a) = f(d) = 0$. By the mean value
theorem, there exists $c\in (a,d)\subset (a,b)$ such that $f'(c) = 0$.
$\Box$

    \textbf{3.} Let $f_k$ be a sequence of bounded (Riemann) integrable
functions defined on $[a,b]$. Suppose $f_k \to f$ uniformly. Then prove
that $f$ is (Riemann) integrable on $[a,b]$, and
\[\int_a^b f_k(x)\,dx \to \int_a^b f(x) \,dx.\]

    \textbf{Proof.} For each $k$, the set $E_k$ of discontinuous points of
$f_k$ has measure zero. Put $E = \bigcup_k E_k$, then $E$ has measure
zero. Outside $E$, $f_k$ is continous for all $k$. So $f$ is continuous
on $[a,b] \backslash E$. This implies that $f$ is integrable on $[a,b]$.

Now, for $\epsilon > 0$, there exists $N> 0$ such that
$\|f_k - f\| \le \epsilon/(b-a)$ for all $k\ge  N$. So we have
\[\left|\int_a^b (f_k(x) - f(x))\,dx\right| \le \int_a^b \left|f_k(x) - f(x)\right|\,dx \le (b-a) \epsilon /(b-a) = \epsilon\]
for all $k > N$. So \[\int_a^b f_k(x)\,dx \to \int_a^b f(x) \,dx. \Box\]

    \textbf{4.} Find functions $f_k :[a,b] \to \mathbb{R}$ which are
integrable and such that $f_k \to f$ pointwise, but $f$ is not
integrable.

    \textbf{Proof.} We may suppose $[a,b] = [0,1]$ for simplicity. Let
$f:[0,1]\to \mathbb{R}$ be defined by $f(0) = 0$ and $f(x) = 1/x$ for
$x\in (0,1]$. For each $k$, put $f_k:[0,1] \to \mathbb{R}$ be defined by
$f(x) = k^2x$ for $x\in [0,1/k)$ and $f(x) =1/x$ for $x\in [1/k,1]$. By
$f_k$ is continuous on $[0,1]$, we have $f_k$ is integrable. Clearly,
$f_k \to f$ pointwise, but $f$ is not integrable on $[0,1]$. $\Box$

    \textbf{5.} Let $f:\mathbb{R} \to \mathbb{R}$ be continuous and set
$F(x) = \int_0^{x^2} f(y)\, dy$. Prove $F'(x) = 2xf(x^2)$.

    \textbf{Proof.} Put $G(x) = \int_0^x f(y)\,dy$, then $F(x) = G(x^2)$. By
$f$ is continuous, we have $G$ is differentiable and $G'(x) = f(x)$. So
the composition of $G$ and the function $(x \mapsto x^2)$ gives us
$F'(x) = 2xG'(x^2) = 2xf(x^2)$. $\Box$

    \textbf{6.} Let $f:[0,1] \to \mathbb{R}$, \[f(x) = \begin{cases}
0, & \text{if } x\in \mathbb{R}\backslash \mathbb{Q},\\
\frac{1}{q}, & \text{if } x = \frac{p}{q},
\end{cases}\] where $p,q\ge 0$ with no common factor. Show $f$ is
integrable and compute $\int_0^1 f$.

    \textbf{Proof.} In HW\#4, Problem 8.b, we already prove that $f$ is
continuous at irrational points and discontinuous at rational points. So
the set of discontinuous points of $f$ has measure zero, which means $f$
is integrable. Now, for any partition $P$ of $[0,1]$, we have
$L(f,P) = 0$, so $\int_{[0,1]}f = 0$. $\Box$

    \textbf{7.} For continuous function $f:\mathbb{R} \to \mathbb{R}$,
define $f_n(x) = n\int_x^{x+1/n}f(\xi)\,d\xi$ for $n=1,2,3,\ldots$ Show
that $df_n(x)/d(x)$ exists even if $df(x)/dx$ does not, and that
$f(x) = \lim_{n\to \infty}f_n(x)$, and that convergence to the limit is
uniform when $f$ is uniformly continuous.

    \textbf{Proof.} Put $g(x) = \int_0^x f(\xi)\,d\xi$, then
$f_n(x) = n(g(x+1/n)-g(x))$ for all $n$. By the fundamental theorem of
caculus, we have $g$ is differentiable, so $f_n$ is differentiable for
all $n$. Moreover, for each $x$, for each $n$, by the mean value
theorem, we have
\[f_n(x)  =n(g(x+1/n) -g(x)] = n.\frac{1}{n}g'(c_{x,n}) = f(c_{x,n}),\]
for some $c_{x,n}\in (x, x+1/n)$. Let $n\to \infty$, we have
$c_{x,n} \to x$, hence $f(c_{x,n}) \to f(x)$. So $f_n(x) \to f(x)$ as
$n\to\infty$. Now, if $f$ is uniformly continuous, then by
$|c_{x,n} - x| < 1/n$ and $1/n$ does not depend on $x$, using a routine
$\epsilon-\delta$ argument, we have $f_n \to f$ uniformly on
$\mathbb{R}$. $\Box$


    % Add a bibliography block to the postdoc
    
    
    
    \end{document}
