
% Default to the notebook output style

    


% Inherit from the specified cell style.




    
\documentclass{article}

    
    
    \usepackage{graphicx} % Used to insert images
    \usepackage{adjustbox} % Used to constrain images to a maximum size 
    \usepackage{color} % Allow colors to be defined
    \usepackage{enumerate} % Needed for markdown enumerations to work
    \usepackage{geometry} % Used to adjust the document margins
    \usepackage{amsmath} % Equations
    \usepackage{amssymb} % Equations
    \usepackage[mathletters]{ucs} % Extended unicode (utf-8) support
    \usepackage[utf8x]{inputenc} % Allow utf-8 characters in the tex document
    \usepackage{fancyvrb} % verbatim replacement that allows latex
    \usepackage{grffile} % extends the file name processing of package graphics 
                         % to support a larger range 
    % The hyperref package gives us a pdf with properly built
    % internal navigation ('pdf bookmarks' for the table of contents,
    % internal cross-reference links, web links for URLs, etc.)
    \usepackage{hyperref}
    \usepackage{longtable} % longtable support required by pandoc >1.10
    \usepackage{booktabs}  % table support for pandoc > 1.12.2
    

    

    
    % Prevent overflowing lines due to hard-to-break entities
    \sloppy 
    % Setup hyperref package
    \hypersetup{
      breaklinks=true,  % so long urls are correctly broken across lines
      colorlinks=true,
      urlcolor=blue,
      linkcolor=darkorange,
      citecolor=darkgreen,
      }
    % Slightly bigger margins than the latex defaults
    
    \geometry{verbose,tmargin=1in,bmargin=1in,lmargin=1in,rmargin=1in}
    
    

    \begin{document}
    
    
    
    
    

    
    \subsection*{Analysis I - homework - week
10}\label{analysis-i---homework---week-10}
\addcontentsline{toc}{subsection}{Analysis I - homework - week 10}

    \textbf{1.} Determine whether the ``curve'' described by the equation
$x^2 + y + \sin(xy) = 0$ can be written in the form $y = f (x)$ in a
neighborhood of $(0, 0)$.

    \textbf{Proof.} Put $F(x,y) = x^2 + y + \sin(xy)$. We have
$\left|\frac{\partial F}{\partial y} (0,0)\right| = 1$. By the Implicit
Function Theorem, we can write $F(x,y) = 0$ by $y = f(x)$ in a
neighborhood of $(0,0)$ for some $f$. $\Box$

    \textbf{2.} Show that the implicit function theorem implies the inverse
function theorem.

    \textbf{Proof.} Let $A \subset \mathbb{R}^n$ be an open set and let
$f: A\subset \mathbb{R}^n \to \mathbb{R}^n$ be of class $C^1$. Let
$x_0\in A$ and suppose $Jf(x_0) \ne 0$. Suppose the Implicit Function
Theorem, we are going to prove that there is a neighborhood $U$ of $x_0$
in $A$ and an open neighborhood $W$ of $f(x_0)$ such that $f(U) = W$ and
$f$ has a $C^1$ inverse $f^{-1}: W \to U$. Moreover, for $y \in W$,
$x = f^{-1}(y)$, we have \[Df^{-1}(y) = [Df(x)]^{-1},\] and if $f$ is of
class $C^{p}$, $p \ge 1$, then so is $f^{-1}$.

Let $F: \mathbb{R}^n \times A \to \mathbb{R}^n$ be defined by
$F(y,x) = f(x) - y$. First, we have $F(f(x_0),x_0) = 0$. It is also
plain to see that $\frac{\partial F}{\partial x}(y,x) = Df (x)$ for all
$(y,x) \in \mathbb{R}^n \times A$. So if $Jf(x_0) \ne 0$, then
$\det \left(\frac{\partial F}{\partial x}(f(x_0),x_0)\right) \ne 0$.
Now, we can apply Implicit Function Theorem to conclude that there
exists an open neighborhood $W$ of $f(x_0)$, an open neighborhood
$U\subset A$ of $x_0$, and an unique function $g:W \to U$ such that
$F(y,g(y)) = 0$ for all $y \in W$. So $f(g(y)) = y$ for all $y \in W$.
Hence $f$ is surjective from $U$ onto $W$. It remains to show that $f$
is one-to-one on $U$. Indeed, suppose $y_0 = f(x_1) = f(x_2)$. Define
the functions $g_1$, $g_2$ from $W$ to $U$ by $g_1(y) = g_2(y) = g(y)$
if $y\ne y_0$ and $g_1(y_0) = x_1$, $g_2(y_0) = x_2$. We have
$F(y,g_1(y)) = F(y,g_2(y) = 0$ for all $y \in W$. By the uniqueness of
$g$, $x_1 = x_2$. So we can conclude $f$ is one-to-one, onto, and
$f^{-1}(y) = g(y)$ on $W$.

Moreover, by Implicit Function Theorem (Corollary 1 in the textbook) and
$\frac{\partial F}{\partial y}(y,x) = -I_n$, if $x=f^{-1}(y)$, then
\[Df^{-1}(y) = Dg(y) = -\left[\frac{\partial F}{\partial x}(y,g(y))\right]^{-1} \frac{\partial F}{\partial y}(y,g(y)) = [Df(g(y))]^{-1} = [Df(x)]^{-1}.\]

Finally, if $f$ is of class $C^{p}$, then $F$ is of class $C^p$, hence
$g$ is of class $C^{p}$ (by Implicit Function Theorem). $\Box$

    \textbf{3.} Is it possible to solve \[\begin{aligned}
xy^2 + xzu + yv^2 &= 3 \\
u^3yz + 2xv − u^2 v^2 &= 2
\end{aligned}\] for $u(x,y,z)$, $v(x,y,z)$ near $(x,y,z) = (1,1,1)$,
$(u,v) = (1,1)$? Compute $\partial v/\partial y$.

    \textbf{Proof.} Put $F_1(x,y,z,u,v) = xy^2 + xzu + yv^2 - 3$,
$F_2(x,y,z,u,v) = u^3yz +2xv - u^2v^2 - 2$, and $F = (F_1, F_2)$. We
have $F(1,1,1,1,1) = 0$ and
\[\frac{\partial F_1}{\partial u}(x,y,z,u,v) = xz, \quad \frac{\partial F_1}{\partial v}(x,y,z,u,v) = 2yv,\]
\[\frac{\partial F_2}{\partial u}(x,y,z,u,v) = 3yzu^2 - 2uv^2, \quad \frac{\partial F_1}{\partial v}(x,y,z,u,v) = 2x - 2u^2v,\]
hence
\[\frac{\partial F_1}{\partial u}(1,1,1,1,1) = 1, \quad \frac{\partial F_1}{\partial v}(1,1,1,1,1) = 2,\]
\[\frac{\partial F_2}{\partial u}(1,1,1,1,1) = 1, \quad \frac{\partial F_1}{\partial v}(1,1,1,1,1) = 0.\]

By $1.0 - 2.1 = -2 \ne 0$, we can apply the Implicit Function Theorem to
solve $F = 0$ for $u(x,y,z)$, $v(x,y,z)$ near $(x,y,z) = (1,1,1)$,
$(u,v) = (1,1)$.

Now, by the implicit function theorem, we have \[
\begin{aligned}
\frac{\partial (u, v)}{\partial (x,y,z)} &= -\left[\frac{\partial F}{\partial (u, v)}\right]^{-1}\frac{\partial F}{\partial (x,y,z)}\\
&= \frac{-1}{xz(2x - 2u^2v) + 2yv(2uv^2 - 3yzu^2)}
\begin{bmatrix}
(2x - 2u^2v) & -2yv \\
(2uv^2-3yzu^2) & xz
\end{bmatrix}\frac{\partial F}{\partial (x,y,z)}.
\end{aligned}\]

Hence, by $\partial F_1/\partial y = 2xy + 2yv$ and
$\partial F_2/\partial y = u^3z$, we have \[
 \begin{aligned}
\frac{\partial v}{\partial y} &= \frac{-1}{xz(2x - 2u^2v) + 2yv(2uv^2 - 3yzu^2)}
\begin{bmatrix}
(2uv^2-3yzu^2) & xz
\end{bmatrix}
\begin{bmatrix}
2xy + v^2 \\
u^3z
\end{bmatrix}\\
&= \frac{(3yzu^2 - 2uv^2)(2xy + v^2) - xz^2u^3}{xz(2x - 2u^2v) + 2yv(2uv^2 - 3yzu^2)}. \Box
\end{aligned}\]

    \textbf{4.} Define the function $F : \mathbb{R}^3 \to \mathbb{R}$ as
follows: \[F (x, y, z) = x^3 z^2 − z^3 yx.\]

Explain why there is no neighborhoods $U$ of $(0, 0)$ and $V$ of $0$
such that there exists an one-to-one function $z = g(x, y)$ defined for
$(x, y) \in U$ and $z \in V$ and satisfying $F (x, y, g(x, y)) = 0$.

    \textbf{Proof.} Suppose on the contrary that there exist such $U$, $V$,
and $g$. Notice that no matter how small $U$ is, there exist
$x_1\neq x_2$ both nonzero so that $(x_1, 0)$ and $(x_2, 0)$ are in $U$.
Put $z_1 = g(x_1, 0)$, $z_2 = g(x_2, 0)$. From the injectivity of $g$,
we have $z_1 \ne z_2$ and from $F(x_1,0,z_1)= F(x_2,0,z_2)= 0$, we have
$x_1^3z_1^2 = x_2^3z_2^2 = 0$. Since $x_1$ and $x_2$ are non-zero, we
have $z_1=z_2$, a contradiction. $\Box$

    \textbf{5.} Let $F : \mathbb{R}^n \to \mathbb{R}^n$ be a continuously
differentiable function. Prove or disprove: If $DF (x)$ is invertible
for every $x \in \mathbb{R}^n$ . Then there exists
$G : F (\mathbb{R}^n ) \to \mathbb{R}^n$ such that $G (F(x)) = x$.

    \textbf{Proof.} Disprove. Let $F: \mathbb{R}^2 \to \mathbb{R}^2$ be
defined by $F(x,y) = (e^x\cos y, e^x \sin y)$. Then
$\det(F(x,y)) = e^x \ne 0$ for all $(x,y) \in \mathbb{R}^2$. So $DF(x)$
is invertible for every $(x,y) \in \mathbb{R}^2$. By
$F(0, 0) = F(0,2\pi)=(1,0)$, $F$ is not one-to-one, hence there doesn't
exist such $G$. $\Box$

    \textbf{6.} Let $f$ be a continuously differentiable mapping of an open
set $E \subset \mathbb{R}^n$ into $\mathbb{R}^n$. Prove that if $Df (x)$
is invertible for every $x\in E$, then $f (W )$ is an open subset of
$\mathbb{R}^n$ for any open set $W \subset E$. In other words, $f$ is an
open mapping of $E$ into $\mathbb{R}^n$.

    \textbf{Proof.} Suppose $Df(x)$ is invertible for every $x\in E$, then
$Jf(x) \ne 0$ for every $x\in E$. Let $W$ be an open subset of $E$ and
let $y_0 \in f(W)$, hence there is $x_0 \in W$ such that
$y_0 = f(x_0) \in f(W)$. By the Inverse Function Theorem, there is a
neighborhood $U_0$ of $x_0$ in $W$ such that $f(U_0)$ is open in
$\mathbb{R}^n$. By $y_0 \in f(U_0) \subset f(W)$ and this is true for
all $y \in f(W)$, we conclude that $f(W)$ is open. $\Box$

    \textbf{7.} $(a)$ Let $f : \mathbb{R}^2 \to \mathbb{R}^2$ be smooth and
suppose that (\emph{Cauchy-Riemann Equations})
\[ \frac{\partial f_1}{\partial x} =  \frac{\partial f_2}{\partial y},\quad  \frac{\partial f_1}{\partial y}=- \frac{\partial f_2}{\partial x}.\]

Show that $Jf (x, y) = 0$ iff $Df (x, y) = 0$; hence $f$ is locally
invertible iff $Df (x, y) \neq 0$. Prove that the inverse function also
satisfies the Cauchy-Riemann equations.

$(b)$ Show that the conclusion of $(a)$ is false (by giving an example)
if $f$ does not satisfy the Cauchy-Riemann equations.

    \textbf{Proof.} $(a)$ If $Df(x,y) = 0$, then trivially $Jf(x,y) = 0$.
Reversely, if $Jf(x,y) = 0$, then by the Cauchy-Riemann equations,
\[Jf(x,y) = \left[\frac{\partial f_1}{\partial x}(x,y)\right]^2 +  \left[\frac{\partial f_1}{\partial y}(x,y)\right]^2 = 0\]
implies $\frac{\partial f_1}{\partial x}(x,y)=0$ and
$\frac{\partial f_1}{\partial y}(x,y)=0$. Hence again, by the
Cauchy-Riemann equations, we conclude that $Df(x,y) = 0$.

Now, if $Df (x, y) \neq 0$, then $Jf(x,y) \ne 0$, hence by the inverse
mapping theorem, $f$ is locally invertible. Conversely, suppose $f$ is
locally invertible. Then from the definition of local invertibility,
$g=f^{-1}$ is differentiable, and by the chain rule,
$Dg(f(x,y)) \circ Df(x,y) = D(g\circ f)(x,y) = D(\operatorname{Id})(x,y) \ne 0$
implies $Df(x,y) \ne 0$.

Now, we prove that $g$ satisfies the Cauchy-Riemann equations. By
Inverse Function Theorem, we have $Dg(z, t) = [Df(g(z, t))]^{-1}$. Now
the right-hand side is of the form
$\begin{bmatrix}a & b \\ -b & a\end{bmatrix}^{-1}$, and hence equals to
$\frac{1}{a^2+b^2}\begin{bmatrix}a & -b \\ b & a\end{bmatrix}$, which is
again of the desired form.

$(b)$ Let $f(x,y) = (x^2,y)$. Then we have
$Df(x,y) = \begin{bmatrix}2x & 0 \\ 0 & 1\end{bmatrix}$ and
$Jf(x,y) = 2x$. So $Jf(0,1) = 0$ but $Df(0,1) \ne 0$. $\Box$


    % Add a bibliography block to the postdoc
    
    
    
    \end{document}
