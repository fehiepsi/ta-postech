
% Default to the notebook output style

    


% Inherit from the specified cell style.




    
\documentclass{article}

    
    
    \usepackage{graphicx} % Used to insert images
    \usepackage{adjustbox} % Used to constrain images to a maximum size 
    \usepackage{color} % Allow colors to be defined
    \usepackage{enumerate} % Needed for markdown enumerations to work
    \usepackage{geometry} % Used to adjust the document margins
    \usepackage{amsmath} % Equations
    \usepackage{amssymb} % Equations
    \usepackage[mathletters]{ucs} % Extended unicode (utf-8) support
    \usepackage[utf8x]{inputenc} % Allow utf-8 characters in the tex document
    \usepackage{fancyvrb} % verbatim replacement that allows latex
    \usepackage{grffile} % extends the file name processing of package graphics 
                         % to support a larger range 
    % The hyperref package gives us a pdf with properly built
    % internal navigation ('pdf bookmarks' for the table of contents,
    % internal cross-reference links, web links for URLs, etc.)
    \usepackage{hyperref}
    \usepackage{longtable} % longtable support required by pandoc >1.10
    

    

    
    % Prevent overflowing lines due to hard-to-break entities
    \sloppy 
    % Setup hyperref package
    \hypersetup{
      breaklinks=true,  % so long urls are correctly broken across lines
      colorlinks=true,
      urlcolor=blue,
      linkcolor=darkorange,
      citecolor=darkgreen,
      }
    % Slightly bigger margins than the latex defaults
    
    \geometry{verbose,tmargin=1in,bmargin=1in,lmargin=1in,rmargin=1in}
    
    

    \begin{document}
    
    
    
    
    

    
    \textbf{2.} (10 points) Let $F_k$ $(k=1,2,3,\ldots)$ be a sequence of
non-empty sets in $\mathbb{R}^n$ with the finite intersection property
(i.e.~any finite subcollection of $F_k$ has a non-empty intersection.)

$(i)$ (5 points) If every $F_k$ is compact, then prove or disprove (by
giving a counterexample) that $\cup_{k=1}^{\infty} F_k$ is non-empty.

$(ii)$ (5 points) If the compact sets in $(i)$ are replaced by closed
sets, does the same conclusion hold? Prove or give a counterexample.

\textbf{Proof.} $(i)$ Suppose on the contrary that
$\cap_{k=1}^{\infty} F_k =\varnothing$. For each $k$, let
$U_k = \mathbb{R}^n \backslash F_k$. Then we have
$\cup_{k=1}^{\infty} U_k = \mathbb{R}^n$. For each $k$, by $F_k$
compact, so closed, we have $U_k$ is open. So $\{U_k\}$ is an open cover
of $\mathbb{R}^n$ and, in particular, of $F_1$. By the compactness of
$F_1$, there exists $k_1,k_2,\ldots,k_n$ such that
$F_1 \subset \cup_{i=1}^n U_{k_i }$ \textbf{(2 points)}. This means that
\[ F_1 \subset \bigcup_{i=1}^n (\mathbb{R}^n \backslash F_{k_i}) = \mathbb{R}^n \backslash \bigcap_{i=1}^n F_{k_i},\]
which means that \textbf{(2 points)}
\[ F_1 \cap \left(\bigcap_{i=1}^n F_{k_i}\right) = \varnothing.\]

We get a contradiction with the finite intersection property of
$\{F_k\}$ \textbf{(1 point)}.

$(ii)$ Counterexample \textbf{(5 points)}: In $\mathbb{R}$, let
$F_k = [k,\infty)$. Then $F_k$ is closed for each $k$ and
$\cap_{k=1}^{\infty} F_k=  \varnothing$. $\Box$

    \textbf{3.} \textbf{(10 points)} A real-valued function $f$ defined on
$(a,b)$ is said to be convex if
\[ f(\lambda x + (1-\lambda)y) \le \lambda f(x) + (1-\lambda)f(y),\]
whenever $a < x < b,a < y < b, 0 < \lambda < 1$. Prove that every convex
function is continuous.

\textbf{Proof.} Let $f$ is a convex function in $(a,b)$. Let
$x\in (a,b)$. It is enough to show that $f$ is right continuous at $x$
(the left continuity is similar). Let $y, z \in (a,b)$ such that
$z < x < y$ and a sequence $\{x_n\}\subset (x,y)$ which converges to
$x$. Let $\lambda_n,\alpha_n\in (0,1)$ be such that
$x_n = \lambda_n x + (1-\lambda_n)y$ and
$x = \alpha_n z + (1-\alpha_n)x_n$. We have $\lambda_n = (y-x_n)/(y-x)$,
hence converges to $1$ as $n\to \infty$; and
$\alpha_n = (x_n-x)/(x_n-z)$, hence converges to $0$ as $n\to \infty$.
By the convexity of $f$, we have \textbf{(4 points)} \[
f(x_n) \le \lambda_n f(x) + (1-\lambda_n)f(y)
\] and \[
f(x) \le \alpha_n f(z) + (1-\alpha_n) f(x_n).
\]

The second inequality is equivalent to \textbf{(2 points)} \[
\frac{f(x) - \alpha_n f(z)}{1-\alpha_n} \le f(x_n).
\]

So we have \textbf{(2 points)} \[
\frac{f(x) - \alpha_n f(z)}{1-\alpha_n} \le f(x_n) \le \lambda_n f(x) + (1-\lambda_n)f(y).
\]

Both the left sequence and the right sequence converges to $f(x)$ as
$n \to\infty$. So $f(x_n) \to f(x)$ as $n\to \infty$ \textbf{(2
points)}. We conclude. $\Box$

    \textbf{6.} \textbf{(10 points)} Let $f_n:[a,b]\to \mathbb{R}$ be
uniformly bounded continuous functions and let
\[F_n(x) = \int_0^x f_n(t)\,dt,\quad a\le x \le b.\]

Prove $F_n$ has a uniformly convergent subsequence.

\textbf{Proof.} Suppose that $|f_n(x)| < M$ for all $x\in [a,b]$, for
all $n$. First, we will show that $\{F_n\}$ is bounded in
$\mathcal{C}([a,b],\mathbb{R}^m)$. Indeed, we have \textbf{(4 points)}
\[\begin{aligned}
|F_n(x)| &= \left| \int_a^x f_n(y) \,dy\right| \\
&\le \int_a^x |f_n(y)|\,dy \le M(x-a)\le M(b-a)
\end{aligned}\] for all $x\in [a,b]$. So $\|F_n\| \le M(b-a)$ for all
$n$.

Now, we will show that $\{F_n\}$ is equicontinous. Indeed, for
$\epsilon > 0$, choose $\delta= \epsilon / M$. Then for all $n$ and for
all $x,y\in [a,b]$ such that $|x-y| < \delta$, we have \textbf{(4
points)} \[\begin{aligned}
|F_n(x) - F_n(y)| &= \left| \int_a^x f_n(t)\, dt - \int_a^y f_n(t)\,dt\right|\\
&= \left|\int_x^y f_n(t)\,dt\right| \le |y-x|M < \epsilon.
\end{aligned}\]

We have $[a,b]$ is compact and
$\{F_n\} \subset \mathcal{C}([a,b] ,\mathbb{R}^m)$ is bounded and
equicontinuous. Applying Arzela-Ascoli's theorem \textbf{(2 points)},
$\{F_n\}$ has a uniformly convergent subsequence. $\Box$


    % Add a bibliography block to the postdoc
    
    
    
    \end{document}
