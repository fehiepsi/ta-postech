
% Default to the notebook output style

    


% Inherit from the specified cell style.




    
\documentclass{article}

    
    
    \usepackage{graphicx} % Used to insert images
    \usepackage{adjustbox} % Used to constrain images to a maximum size 
    \usepackage{color} % Allow colors to be defined
    \usepackage{enumerate} % Needed for markdown enumerations to work
    \usepackage{geometry} % Used to adjust the document margins
    \usepackage{amsmath} % Equations
    \usepackage{amssymb} % Equations
    \usepackage[mathletters]{ucs} % Extended unicode (utf-8) support
    \usepackage[utf8x]{inputenc} % Allow utf-8 characters in the tex document
    \usepackage{fancyvrb} % verbatim replacement that allows latex
    \usepackage{grffile} % extends the file name processing of package graphics 
                         % to support a larger range 
    % The hyperref package gives us a pdf with properly built
    % internal navigation ('pdf bookmarks' for the table of contents,
    % internal cross-reference links, web links for URLs, etc.)
    \usepackage{hyperref}
    \usepackage{longtable} % longtable support required by pandoc >1.10
    

    

    
    % Prevent overflowing lines due to hard-to-break entities
    \sloppy 
    % Setup hyperref package
    \hypersetup{
      breaklinks=true,  % so long urls are correctly broken across lines
      colorlinks=true,
      urlcolor=blue,
      linkcolor=darkorange,
      citecolor=darkgreen,
      }
    % Slightly bigger margins than the latex defaults
    
    \geometry{verbose,tmargin=1in,bmargin=1in,lmargin=1in,rmargin=1in}
    
    

    \begin{document}
    
    
    
    
    

    
    \emph{Note.} In sketch of HW\#7: Exercise 1.a, misplace of
$(\Rightarrow)$ and $(\Leftarrow)$. Exercise 6.a, logarithm is a concave
function. Exercise 7, consider $f: [1,\infty)\to [1,\infty)$ with
$f(x) = x + 1/x$.

    \subsection*{Analysis I - homework - week
8}\label{analysis-i---homework---week-8}
\addcontentsline{toc}{subsection}{Analysis I - homework - week 8}

    \textbf{1.} Let $f: X\to \mathbb{R}$ be a function and define
$\operatorname{supp}(f)$, the support of $f$, as the closure of subset
of $X$ where $f$ is non-zero, i.e.;
\[ \operatorname{supp}(f) := \operatorname{cl}(\{x\in X \mid f(x) \neq 0\}).\]

Define
\[ \mathcal{C}_0(\mathbb{R},\mathbb{R}) := \{f:\mathbb{R} \to \mathbb{R} \mid \operatorname{supp}(f) \text{ is compact}\} \cap \mathcal{C}(\mathbb{R},\mathbb{R}).\]

Prove that $\mathcal{C}_0(\mathbb{R},\mathbb{R})$ is algebra.

\textbf{Sketch of proof.} It is enough to check that if
$f,g \in \mathcal{C}_0(\mathbb{R},\mathbb{R})$, $\alpha \in \mathbb{R}$
then $f+g$, $fg$, $\alpha f$ in $\mathcal{C}(\mathbb{R},\mathbb{R})$.

    \textbf{2.} Let $f$ be a continuous function from $[0,1]$ to
$\mathbb{R}$.
\[\int_0^1 f(x)x^n\,dx = 0, \quad \text{for all } n \in \mathbb{N} \cup \{0\},\]
then $f(x) = 0$ for all $x\in [0,1]$.

\textbf{Sketch of proof.} By assumption, we have
\[ \int_0^1 f(x) p(x) \,dx = 0,\] for $p$ is any polynomial in
$\mathbb{R}$. Use Stone-Weierstrass theorem to show that
\[\int_0^1 f^2(x) \,dx = 0.\]

    \textbf{3.} Let $f$ be a continuous function form $[0,1]$ to
$\mathbb{R}$. Prove that if
\[\int_0^1 f(x)e^{nx}\,dx = 0, \quad \text{for all } n \in \mathbb{N} \cup \{0\},\]
then $f(x) = 0$ for all $x\in [0,1]$.

\textbf{Proof.} Similar to Exercise 2, we just need to prove that the
set $\mathcal{B}$ of finite linear combinations of $e^{nx}$ where
$n \in \mathbb{N} \cup \{0\}$ is dense in
$\mathcal{C}([0,1], \mathbb{R})$. Use Stone-Weierstrass theorem (Theorem
5.12).

    \textbf{4.} Prove that $f(x) = 1/x$ cannot be approximated uniformly by
polynomials on $(0,1)$, i.e., there is no sequence of polynomial $f_n$
such that $f_n \to f$ uniformly on $(0,1)$.

\textbf{Sketch of proof.} Every polynomial is bounded on $[0,1]$. If $f$
can be approximated by a polynomial, then $f$ is bounded on $(0,1)$. Get
a contradiction from here.

    \textbf{5.} Let $f:\mathbb{R}^2 \to \mathbb{R}^2$,
$f(x,y) := (x^2,y^2)$. Prove that
\[Df(x,y) = \begin{pmatrix}2x & 0\\0 & 2y\end{pmatrix}\] directly form
the definition in the text book.

\textbf{Sketch of proof.} Use definition.


    % Add a bibliography block to the postdoc
    
    
    
    \end{document}
