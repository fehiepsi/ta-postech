
% Default to the notebook output style

    


% Inherit from the specified cell style.




    
\documentclass{article}

    
    
    \usepackage{graphicx} % Used to insert images
    \usepackage{adjustbox} % Used to constrain images to a maximum size 
    \usepackage{color} % Allow colors to be defined
    \usepackage{enumerate} % Needed for markdown enumerations to work
    \usepackage{geometry} % Used to adjust the document margins
    \usepackage{amsmath} % Equations
    \usepackage{amssymb} % Equations
    \usepackage[mathletters]{ucs} % Extended unicode (utf-8) support
    \usepackage[utf8x]{inputenc} % Allow utf-8 characters in the tex document
    \usepackage{fancyvrb} % verbatim replacement that allows latex
    \usepackage{grffile} % extends the file name processing of package graphics 
                         % to support a larger range 
    % The hyperref package gives us a pdf with properly built
    % internal navigation ('pdf bookmarks' for the table of contents,
    % internal cross-reference links, web links for URLs, etc.)
    \usepackage{hyperref}
    \usepackage{longtable} % longtable support required by pandoc >1.10
    \usepackage{booktabs}  % table support for pandoc > 1.12.2
    

    

    
    % Prevent overflowing lines due to hard-to-break entities
    \sloppy 
    % Setup hyperref package
    \hypersetup{
      breaklinks=true,  % so long urls are correctly broken across lines
      colorlinks=true,
      urlcolor=blue,
      linkcolor=darkorange,
      citecolor=darkgreen,
      }
    % Slightly bigger margins than the latex defaults
    
    \geometry{verbose,tmargin=1in,bmargin=1in,lmargin=1in,rmargin=1in}
    
    

    \begin{document}
    
    
    
    
    

    
    \subsection*{Analysis I - homework - week 15
(HW\#14)}\label{analysis-i---homework---week-15-hw14}
\addcontentsline{toc}{subsection}{Analysis I - homework - week 15
(HW\#14)}

    \textbf{1.} $(a)$ Let $f : [a, b] \to \mathbb{R}$ be differentiable and
assume that $f'$ is integrable. Prove
$\int_a^b f'(x)\,dx = f (b) − f(a)$.

$(b)$ Must $f'$ always be integrable?

    \textbf{Sketch of proof.} $(a)$ See the lemma in the proof of Problem 1,
HW\#13.

$(b)$ No. Consider the function $f:[0,1]\to \mathbb{R}$ be defined by
$f(x) = x^2 \sin(x^{-2}) $ for $0 < x \le 1$ and $f(0) = 0$. Then
$f'(x) = 2x\sin(x^{-2}) -2\cos (x^{-1})/x$ if $x>0$ and
\[f'(0) = \lim_{x\to 0^+}\frac{ x^2\sin(x^{-2}) - 0}{x} =  \lim_{x\to 0^+} x\sin (x^{-2}) = 0\]
by $\sin (x^{-2}) \le 1$ for any $x > 0$. Show that this derivative is
unbounded in $(0,1)$, so it is not Riemann integrable. $\Box$

\emph{Note.} If we regard integrable by improper integrable, we still
have a counterexample for $(b)$ (e.g.~Volterra's function).

    \textbf{2.} For $x > 0$ define $L(x) = \int_1^x 1/t \,dt$. Prove the
following, using this definition,

$(a)$ $L$ is increasing in $x$.

$(b)$ $L(xy) = L(x) + L(y)$.

$(c)$ $L'(x) = 1/x$.

$(d)$ $L(1) = 0$.

$(e)$ Properties $(c)$ and $(d)$ uniquely determine $L$.

    \textbf{Sketch of proof.} This is a routine exercise. For $(b)$, change
of variables. $\Box$

    \textbf{3.} Suppose $f : (0, b] \to \mathbb{R}$ is continuous, positive,
and integrable on $(0, b]$. Suppose further that as $x \to 0$ from the
right, $f (x)$ increases monotonically to $+\infty$. Then prove that
$\varepsilon f (\varepsilon) \to 0$ as $\varepsilon \to 0$.

    \textbf{Sketch of proof.} Show that
$\lim_{x\to 0^+} \int_0^x f(t)\,dt = 0$. By $f(x)$ increases
monotonically to $+\infty$ as $x\to 0^+$, we have
$\int_0^x f(t)\,dt \ge x.f(x)$ for any $x > 0$. Let $x \to 0^+$, we get
$xf(x) \to 0$. $\Box$

    \textbf{4.} Show that $\int_1^{\infty} x^{-p}\sin x \, dx$ converges if
$p > 1$. Show that if $0 < p ≤ 1$, then the convergence is conditional.

    \textbf{Sketch of proof.} By $|\sin x | \le 1$, we have
$|x^{-p}\sin x| \le x^{-p}$. By $\int_1^{\infty} x^{-p}\,dx = 1/(p-1)$
if $p > 1$, we conclude that $\int_1^{\infty}x^{-p}\sin x \,dx$
converges if $p > 1$.

For $0 < p \le 1$, use integration by parts, we get
\[\int_1^R x^{-p}\sin x \,dx =\left. \frac{-\cos x}{x^p}\right|^{x=R}_{x= 1} - \int_1^R \frac{p\cos x}{x^{p+1}}\,dx= \frac{-\cos R}{R^p} + \cos 1 - p \int_1^R \frac{\cos x }{x^{p+1}}\,dx.\]

Show that $\int_1^R \cos x /x^{p+1}\,dx$ converges as $R\to +\infty$.
Conclude that $\int_1^R x^{-p}\sin x \,dx$ converges conditionally.
$\Box$

    \textbf{5.} The gamma function is defined by the improper integral
$\Gamma (p) = \int_0^{\infty} e^{-x} x^{p-1} \,dx$. Show that the
integral is convergent for $p > 0$.

    \textbf{Sketch of proof.} We break the above integral into two parts
$I_1 = \int_0^{1} e^{-x} x^{p-1} \,dx$ and
$I_2 = \int_1^{\infty} e^{-x} x^{p-1} \,dx$.

For the first integral, we bound
$I_1 \le  \int_0^{1} x^{p-1} \,dx = 1/p$.

For the second integral, we use the fact that
$\lim_{x\to \infty } x^{p-1}/e^{x/2} = 0$ to conclude that there is a
constant $C > 0$ such that $x^{p-1}\le Ce^{x/2}$ for all
$x\in [1, \infty)$. This allows us to bound
$I_2\le \int_1^{\infty} Ce^{-x/2} \,dx= 2Ce^{-1/2}$. $\Box$

\emph{Note.} The trick in the second integral is very useful when we
want to estimate some integral.

    \textbf{6.} Let
$R([0, 1]) = \{f : [0, 1] → \mathbb{R} \mid f \text{ is Riemann integrable}\}$.
Set \[d(f,g) = \int_0^1 |f (x) − g(x)|\,dx.\]

Is $d$ a metric on the space $R([0, 1])$?

    \textbf{Sketch of proof.} $d$ is not a metric. Let $f$ be the function
that equals 1 at 0 and equals 0 otherwise, and let $g = 2f$, then
$f \neq g$ but $d(f, g) = 0$. $\Box$

    \textbf{7.} Is $\int_0^{\infty}x^p \,dx$ convergent for any $p$? If so,
which $p$?

    \textbf{Sketch of proof.} The integral is not convergent for any $p$.
Calculate $\int_{\varepsilon}^{N}x^p \,dx$, then let $\epsilon \to 0$ or
$N\to +\infty$ to prove this claim. $\Box$

    \textbf{8.} Let $f : [0, 1] → \mathbb{R}$ be integrable and be
continuous at $x_0$. Show that the map $I_f (x) = \int_0^x f (y)\, dy$
is differentiable with derivative $f (x_0 )$. Give an example of a
discontinuous integrable $f$ for which $I_f$ is not differentiable. For
bounded integrable $f$ prove this map is always continuous.

    \textbf{Sketch of proof.} Assume that $f$ is integrable and be
continuous at $x_0$. Then for every $\epsilon>0$, there exists a
$\delta>0$ such that $|f(t)-f(x)|\le \epsilon$ whenever
$|t-x|\le \delta$. Then for every
$y\in [0, 1]\cap [x_0-\delta, x_0+\delta]\setminus \{x_0\}$, one has
\[\left|\frac{I_f(y) - I_f(x_0)}{y-x_0}-f(x_0)\right| \le \frac{1}{|y-x_0|}\left|\int_{x_0}^{y} (f(t) - f(x_0))\,dt\right|\le \epsilon.\]

For the second part, let $f(x)  = \chi_{[0, 1/2]}(x)$. One has
$I_f(x) = x$ if $x \le 1/2$ and $I_f(x) = 1/2$ if $x > 1/2$. This
function is not differentiable at $1/2$ (why?).

For the last part, assume that $|f|\le M$ for some $M > 0$. Then for
every $\varepsilon>0$, if $|y-x|\le \varepsilon / M$ then
\[|I_f(y) - I_f(x)|\le \left|\int_{x}^y  f(t)\, dt \right| \le |y-x|M \le \varepsilon. \Box\]

    \textbf{9.} Prove that $\lim_{n\to\infty} (n!)^{1/n}/n = e^{−1}$ by
considering Riemann sums for $\int_0^1 \log x\,dx$ based on the
partition $1/n < 2/n < \cdots < 1$.

    \textbf{Sketch of proof.} Instead of considering the integral
$\int_0^1 \log x \, dx$, we consider the integral
$\int_1^{\infty} \log x \,dx$ based on the partition
$1 < 2 < \cdots < n$. Since the function $x\to \log x$ is monotone
increasing on the interval $[1, \infty)$, one has
\[\int_{1}^{n}\log x\,dx = \sum_{i=2}^{n}\int_{i-1}^{i}\log x\, dx \le \sum_{i=2}^{n}\int_{i-1}^{i}\log i\, dx=\sum_{i=2}^{n}\log i = \log(n!).
\]

Similarly, \[\log (n!)\le \int_{2}^{n+1}\log x\,dx.\]

Thus,
\[\int_{1}^{n}\log x\,dx\le \log (n!)\le \int_{2}^{n+1}\log x\,dx.\]

Show that
\[-1\le \frac{\log(n!)}{n}-\log n\le -1 +\log\frac{n+1}{n}+\frac{\log (n+1)}{n}-\frac{2\log 2-1}{n}.\]

Letting $n\to \infty$ to get the solution. $\Box$

    \textbf{10.} It is a fact that
\[\sin\left(\frac{\pi}{n}\right)\sin\left(\frac{2\pi}{n}\right)\ldots \sin\left(\frac{(n-1)\pi}{n}\right) = \frac{n}{2^{n-1}}.\]

Use this identity to evaluate $\int_0^{\pi}\log \sin x\,dx$. {[}You
don't need to prove the identity.{]}

    \textbf{Sketch of proof.} Note that the function $\log \sin x$ is
integrable on $(0,\pi/2)$. Indeed, for $x \in (0, \pi/2)$, by
$2x/\pi <  \sin x$ (this is an elementary inequality, to prove, just
consider the function $(\sin x - 2x/\pi)$), we have
\[|\log\sin x| = - \log \sin x = \log \left(\frac{1}{\sin x}\right) < \log \left (\frac{\pi}{2 x}\right) =  \log(\pi/2) - \log (x).\]

Claim that $\log\sin x$ is integrable on $(0,\pi/2)$. By change of
variables from $x \mapsto (\pi - x)$, we get $\log \sin x$ integrable on
$(\pi/2,\pi)$, hence $\log\sin x$ is integrable on $(0,\pi)$.

Let $I=\int_0^{\pi}\log \sin x\,dx$. We don't need the identity in the
question to evaluate the integral. Using the formula
$\sin x = 2\sin \frac{x}{2}\cos\frac{x}{2}$, one gets,
\[I =\pi \log 2 + 2\int_0^{\pi/2}\log \sin x\,dx + 2\int_0^{\pi/2}\log \cos x\,dx.\]
Now, the two last integrals are equal because $\sin x = \cos(\pi/2-x)$;
and each of them equals $I/2$. This implies $I = -\pi\log 2$. $\Box$


    % Add a bibliography block to the postdoc
    
    
    
    \end{document}
