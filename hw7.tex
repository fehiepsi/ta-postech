
% Default to the notebook output style

    


% Inherit from the specified cell style.




    
\documentclass{article}

    
    
    \usepackage{graphicx} % Used to insert images
    \usepackage{adjustbox} % Used to constrain images to a maximum size 
    \usepackage{color} % Allow colors to be defined
    \usepackage{enumerate} % Needed for markdown enumerations to work
    \usepackage{geometry} % Used to adjust the document margins
    \usepackage{amsmath} % Equations
    \usepackage{amssymb} % Equations
    \usepackage[mathletters]{ucs} % Extended unicode (utf-8) support
    \usepackage[utf8x]{inputenc} % Allow utf-8 characters in the tex document
    \usepackage{fancyvrb} % verbatim replacement that allows latex
    \usepackage{grffile} % extends the file name processing of package graphics 
                         % to support a larger range 
    % The hyperref package gives us a pdf with properly built
    % internal navigation ('pdf bookmarks' for the table of contents,
    % internal cross-reference links, web links for URLs, etc.)
    \usepackage{hyperref}
    \usepackage{longtable} % longtable support required by pandoc >1.10
    

    

    
    % Prevent overflowing lines due to hard-to-break entities
    \sloppy 
    % Setup hyperref package
    \hypersetup{
      breaklinks=true,  % so long urls are correctly broken across lines
      colorlinks=true,
      urlcolor=blue,
      linkcolor=darkorange,
      citecolor=darkgreen,
      }
    % Slightly bigger margins than the latex defaults
    
    \geometry{verbose,tmargin=1in,bmargin=1in,lmargin=1in,rmargin=1in}
    
    

    \begin{document}
    
    
    
    
    

    
    \subsection*{Analysis I - homework - week
7}\label{analysis-i---homework---week-7}
\addcontentsline{toc}{subsection}{Analysis I - homework - week 7}

    \textbf{1.} $(a)$ Prove that if $A\subset \mathbb{R}^n$ is compact,
$B \subset\mathcal{C}(A, \mathbb{R}^m)$ is compact $\Leftrightarrow$ $B$
is closed, bounded, and equicontinuous.

$(b)$ Let $D = \{f \in\mathcal{C}([0,1], \mathbb{R}) \mid \|f\|\le 1\}$.
Show $D$ is closed and bounded, but is not compact.

\textbf{Proof.} $(a)$ For the $(\Rightarrow)$ side, because $B$ is
compact, we have $B$ is closed and bounded. To show the equicontinuity,
use $\epsilon/3$ trick. First, fix $a\in A$. For $\epsilon >0$,
$B \subset D(f_1,\epsilon/3)\cup \ldots \cup D(f_n,\epsilon/3)$ for some
$f_1,\ldots,f_n$ by the compactness of $B$. Because each $f_i$ is
continuous at $a$, there exists $\delta > 0$ such that
$|f_i(x) - f_i(a)| < \epsilon/3$ for all $x\in D(a,\delta)$, for
$1\le i \le n$. Let $f \in B$, there exists $i$ ($1\le i \le n$) such
that $\|f - f_i\| < \epsilon/3$. So
\[|f(x) - f(a)| \le |f(x) - f_i(x)| + |f_i(x) - f_i(a)| + |f_i(a) - f(a)| < \epsilon\]
for all $x\in D(a,\delta)$. This implies that $B$ is equicontinuous.

For the $(\Leftarrow)$ side, by Arzela-Ascoli theorem, each sequence in
$B$ has a subsequence converges uniformly. By $B$ is closed, this
subsequence must converges uniformly to a function in $B$. So $B$ is
compact.

$(b)$ Clearly $D$ is closed and bounded. To show that $D$ is not
compact, let $f_n(x) = x^n$, then $\{f_n\}$ cannot have a subsequence
converges uniformly (on the contrary, the limit $g$ of that subsequence
is continuous, but $\lim_{n\to\infty}x^n = 1$ if $x =1$ implies
$g(1) =1$ and $\lim_{n\to\infty}x^n = 0$ if $x < 1$ implies $g(x) = 0$
if $x < 1$, hence $g$ is not continuous at $1$). $\Box$

    \textbf{2.} Suppose $f:\mathbb{R}\to\mathbb{R}$ is continuous and
$f(1) = 7$. Suppose $f(x)$ is rational for all $x$. Prove $f$ is
constant.

\textbf{Proof.} If $x < 1$ and $f(x) < 7$, then by $[x,1]$ is connected,
we have $f([x,1])$ is connected. So we have
$[f(x) , 7] \subset f([x,1])$, which implies that there exists
$y\in [x,1]$ such that $f(y)$ is irrational. Similarly for other cases,
we conclude that $f$ is constant. $\Box$

    \textbf{3.} Let $\{f_n\}$ be a uniform convergent sequence of uniform
continuous functions. Prove that $\{f_n\}$ is equicontinuous.

\textbf{Proof.} In HW\#6, Exercise 2, we now that $f_n$ converges to a
(uniformly) continuous function $f$. Fix a point $a \in X$. Using
$\epsilon/3$ trick again, for $\epsilon > 0$, there exists $N$ such that
$d(f_n(x),f(x)) < \epsilon/3$ for all $x\in X$, for all $n > N$. Choose
$\delta_1 > 0$ such that $d(f(x),f(a)) < \epsilon /3$ if
$d(x,a) < \delta_1$. For $n > N$ and $x\in D(a,\delta_1)$, we have
\[d(f_n(x),f_n(a))\le d(f_n(x) ,f(x))+d(f(x),f(a))+d(f(a),f_n(a)) < \epsilon.\]
By $\{f_1,f_2,...,f_N\}$ is a finite set of continuous functions at $a$,
there exists $\delta_2>0$ such that \[d(f_n(x),f_n(a)) < \epsilon\] for
all $x\in D(a,\delta_2)$, for all $n\le N$. Put
$\delta = \min\{\delta_1,\delta_2\}$, we have
$d(f_n(x),f_n(a)) < \epsilon$ for all $x\in D(x,\delta)$, for all $n$.
So $\{f_n\}$ is equicontinous. $\Box$

\emph{Note.} The assumption that each $f_n$ is uniform continuous is not
necessary, continuous is enough. If each $f_n$ is uniformly continous, a
similar argument can show that $\{f_n\}$ is uniformly equicontinous.

    \textbf{4.} Let $f:\mathbb{R}\to\mathbb{R}$ be a uniform limit of
polynomials. Prove that $f$ is a polynomial.

\textbf{Proof.} Let $\{p_n\}$ be a sequence of polynomials which
converges uniformly to $f$. Let $\epsilon = 1$, there exists $N$ such
that $\|p_n(x) - p_m(x)\| < \epsilon$ for all $x\in \mathbb{R}$, for all
$n > m \ge N$. Let $q_n = p_n - p_N$, then $q_n$ is bounded for all
$n\ge N$. We claim that if a polynomial $q$ is bounded then it is
constant. Indeed, suppose $\deg(g) > 1$, we may write
$q(x) = a_kx^k + a_{k-1}x^{k-1} +\ldots + a_0$ where $a_k \ne 0$ and
$k\ge 1$. We have
\[q(x) = a_kx^k(1+\frac{a_{k-1}}{a_k}x^{-1} + \ldots + \frac{a_0}{a_l}x^{-k}).\]

So we have $q(x) \to +\infty$ as $x \to +\infty$ if $a_k > 0$ and
$q(x) \to -\infty$ as $x \to +\infty$ if $a_k < 0$. We get a
contradiction.

Our claim implies that $q_n$ is constant for all $n\ge N$. Put
$c_n = q_n(0)$. By $\{q_n\}$ converges uniformly to $f-p_N$, we have
$\{c_n\}$ converges uniformly to $f-p_N$. So $f-p_N$ is a constant,
which is $f(0) - p_N(0)$. So $f = p_N + f(0) - p_N(0)$, which is a
polynomial. $\Box$

    \textbf{5.} Prove that if

$(a)$ $f_n$, $g(x)$ continuous, $0 \le x < \infty$,

$(b)$ $|f_n| < g(x)$, $n = 1,2,\ldots$, $0 \le x < \infty$,

$(c)$ $f_n(x) \to f(x)$ uniformly, $0 \le x \le R$, for any
$R < \infty$,

$(d)$ $\int_0^{\infty}g(x)\,dx < \infty$,

then
\[\lim_{n\to\infty}\int_0^{\infty}f_n(x)\,dx= \int_0^{\infty}f(x)\,dx.\]

\textbf{Proof.} Using $\epsilon/3$ trick, let $\epsilon > 0$. By $(d)$,
there is $R>0$ such that $\int_R^{\infty} g(x)\,dx < \epsilon/3$. We
have $f_n(x) \to f(x)$ uniformly on $[0,R]$, so there exists $N$ such
that $|f_n(x) - f(x)| < \epsilon/3R$ for all $n\ge N$. So for $n\ge N$,
we have $\int_0^R|f_n(x) - f(x)| \,dx \le \epsilon/3$. By $(b)$ and
$(c)$, we have $|f(x)| \le g(x)$ for all $x\in [0,\infty)$, so
$\int_R^{\infty}| f_n(x)|\,dx < \epsilon/3$ and
$\int_R^{\infty} |f(x)|\,dx < \epsilon/3$, hence
$\int_R^{\infty} | f_n(x) -f(x)|\,dx < 2\epsilon /3$, if $n \ge N$.
Totally, we have $\int_0^{\infty} |f_n(x) - f(x)| < \epsilon$ for all
$n\ge N$. This implies that
\[\lim_{n\to\infty}\int_0^{\infty}f_n(x)\,dx= \int_0^{\infty}f(x)\,dx.\Box\]

    \textbf{6.} $(a)$ \emph{Young's inequality:} Let $p>1$ with $1/p+1/q=1$.
For $a,b,t>0$, prove that \[ab\le \frac{a^pt^p}{p}+\frac{b^qt^{-q}}{q}\]
and that $ab$ is the minimum value of the right side.

$(b)$ \emph{Holder's inequality:} Let $a_k,b_k\ge 0$ and $p > 1$, and
$1/p + 1/q = 1$. Prove that
\[\sum_1^n a_kb_k\le \left(\sum_1^n a_k^p\right)^{1/p} \left(\sum_1^n b_k^q\right)^{1/q}.\]

$(c)$ \emph{Minkowski's inequality:} Let $a_k,b_k\ge 0$ and $p > 1$.
Prove that
\[\left(\sum_1^n (a_k+b_k)^p\right)^{1/p}\le \left(\sum_1^n a_k^p\right)^{1/p} + \left(\sum_1^n b_k^p\right)^{1/p}.\]

\textbf{Proof.} $(a)$ Put $c=at$, $d=bt^{-1}$, we want to prove that
\[ cd \le \frac{c^p}{p} + \frac{d^q}{q}.\]

By the second derivative of $e^x$ on $(0,\infty)$ is positive, we have
the function $e^x$ is convex on $(0,\infty)$. So we have
\[\begin{aligned}
cd &= \exp(\ln(cd)) = \exp(\ln(c) + \ln(d))\\
&= \exp\left(\frac{1}{p}\ln(c^p) + \frac{1}{q}\ln(d^q)\right) \\
& \le \frac{1}{p}\exp(\ln(c^p)) +\frac{1}{q}\exp(\ln(d^q)) \\
&= \frac{c^p}{p} + \frac{d^q}{q}.
\end{aligned}\]

The above inequality comes from the convexity of $e^x$ and
$1/p + 1/q = 1$. The right side in the question attains its minimum $ab$
when $t = a^{-p/(p+q)} b^{q/(p+q)}$.

$(b)$ Excluding the trivial cases (all $a_k$ is $0$ or all $b_k$ is
$0$), put $c_k = a_k/\left(\sum_{1}^n a_k^p\right)^{1/p}$,
$d_k = b_k/\left(\sum_{1}^n b_k^q\right)^{1/q}$. Then
$\sum_{1}^n c_k^p = \sum_{1}^n d_k^q = 1$. By $(a)$, we have
\[\begin{aligned}
1  &=\frac{1}{p} + \frac{1}{q} = \frac{\sum_{1}^n c_k^p}{p} + \frac{\sum_{1}^n d_k^q }{q} = \sum_{1}^n \left(\frac{c_k^p}{p} + \frac{d_k^q}{q}\right)\\
&\ge \sum_{1}^n c_k d_k  = \sum_1^n \left(\frac{a_k}{\left(\sum_{1}^n a_k^p\right)^{1/p}}\cdot\frac{b_k}{\left(\sum_{1}^n b_k^q\right)^{1/q}}\right).
\end{aligned}\]

We conclude.

$(c)$ We have
\[\sum_1^n(a_k+b_k)^p = \sum_1^n(a_k+b_k)^{p-1}a_k + \sum_1^n(a_k+b_k)^{p-1}b_k,\]
and by $(b)$ \[\begin{aligned}
\sum_1^n(a_k+b_k)^{p-1}a_k &\le \left(\sum_1^n (a_k+b_k)^{(p-1)q}\right)^{1/q}\left(\sum_1^n a_k^p\right)^{1/p}\\
&= \left(\sum_1^n (a_k+b_k)^p\right)^{1/q}\left(\sum_1^n a_k^p\right)^{1/p},\\
\sum_1^n(a_k+b_k)^{p-1}b_k &\le \left(\sum_1^n (a_k+b_k)^{(p-1)q}\right)^{1/q}\left(\sum_1^n b_k^p\right)^{1/p}\\
&= \left(\sum_1^n (a_k+b_k)^{p}\right)^{1/q}\left(\sum_1^n b_k^p\right)^{1/p}.
\end{aligned}\]

So
\[ \sum_1^n(a_k+b_k)^p \le \left(\sum_1^n (a_k+b_k)^{p}\right)^{1/q} \left( \left(\sum_1^n a_k^p\right)^{1/p} + \left(\sum_1^n b_k^p\right)^{1/p}\right),\]
which implies
\[\left(\sum_1^n (a_k+b_k)^p\right)^{1/p}\le \left(\sum_1^n a_k^p\right)^{1/p} + \left(\sum_1^n b_k^p\right)^{1/p}.\Box \]

    \textbf{7.} Let $f:X\to X$ be a continuous function, where $X$ is a
complete metric space (such as $\mathbb{R}$) satisfying
$d(f(x), f(y)) < d(x, y)$ for all $x\ne y\in X$. Must $f$ have a fixed
point? Discuss. What if $X$ is compact?

\textbf{Proof.} If $X$ is just complete, the answer is no. Consider
$f:[1,\infty) \to [1,\infty)$ with $x\mapsto x + 1/x$. We have
\[ f(x)- f(y) = x+ \frac{1}{x} - y - \frac{1}{y} = (x-y)\left(1- \frac{1}{xy}\right).\]

So $|f(x) - f(y)| < |x-y|$ if $x\ne y$ (in this case, we always have
$xy > 1$). But $f$ has no fixed point because $f(x) > x$ for all
$x\in [1,\infty)$.

If $X$ is compact, the answer is yes. Consider the function
$\varphi : X\to\mathbb{R}$ with $x\mapsto d(f(x),x)$. By $f$ and $d$ are
continous functions, we have $\varphi$ is continuous. By $X$ is compact,
we have $\varphi(X)$ is also compact, so it has a minimum $a\ge 0$. If
$a > 0$, then $d(f(x),x) > 0$, hence $f(x) \ne x$, for all $x\in X$.
Moreover, we have $\varphi(f(x)) = d(f(f(x)),f(x)) < d(f(x),x) = a$,
which contradicts to $a = \min \varphi(X)$. So $a= 0$, which means that
there exists $x\in X$ such that $d(x,f(x)) = a= 0$. In other words,
$f(x) = x$ and $x$ is a fixed point. $\Box$

    \textbf{8.} $(a)$ Define
$I : \mathcal{C}([0,1], \mathbb{R}) \to \mathbb{R}$ as follow:

\[I(f) =\int_0^1 f(x)\,dx.\]

Prove that $I$ is continuous.

$(b)$ Show that

\[\left\{ f\in\mathcal{C}([0,1], \mathbb{R}) \mid \int_0^1 f(x) \, dx \in (0,3)\right\} \]
is open.

\textbf{Proof.} $(a)$ We have
\[| I(f) - I(g) | = \left|\int_0^1 (f(x) - g(x)) \,dx \right| \le \|f - g\|.\]

So $I$ is a Lipschitz continuous function, hence it is continuous.

$(b)$ The set in the question is the pre-image of the open set $(0,3)$
under $I$. Because $I$ is continous, that set must be open. $\Box$

    \textbf{9.} Let $f_n:[a,b]\to \mathbb{R}$ be uniformly bounded
continuous functions. Set
\[F_n(x) = \int_0^x f_n(t)\,dt,\qquad a\le x \le b.\]

Prove $F_n$ has a uniformly convergent subsequence.

\textbf{Proof.} Suppose that $|f_n(x)| < M$ for all $x\in [a,b]$, for
all $n$. First, we will show that $\{F_n\}$ is bounded in
$\mathcal{C}([a,b],\mathbb{R}^m)$. Indeed, we have \[\begin{aligned}
|F_n(x)| &= \left| \int_a^x f_n(y) \,dy\right| \\
&\le \int_a^x |f_n(y)|\,dy \le M(x-a)\le M(b-a)
\end{aligned}\] for all $x\in [a,b]$. So $\|F_n\| \le M(b-a)$ for all
$n$.

Now, we will show that $\{F_n\}$ is equicontinous. Indeed, for
$\epsilon > 0$, choose $\delta= \epsilon / M$. Then for all $n$ and for
all $x,y\in [a,b]$ such that $|x-y| < \delta$, we have \[\begin{aligned}
|F_n(x) - F_n(y)| &= \left| \int_a^x f_n(t)\, dt - \int_a^y f_n(t)\,dt\right|\\
&= \left|\int_x^y f_n(t)\,dt\right| \le |y-x|M < \epsilon.
\end{aligned}\]

We have $[a,b]$ is compact and
$\{F_n\} \subset \mathcal{C}([a,b] ,\mathbb{R}^m)$ is bounded and
equicontinuous. Applying Arzela-Ascoli's theorem, $\{F_n\}$ has a
uniformly convergent subsequence. $\Box$

    \textbf{10.} Let
$T : \mathcal{C}_b([0, r], \mathbb{R}) \to \mathcal{C}_b([0, r], \mathbb{R})$
be defined by \[ T(f)(x) = \alpha f(x) +\int_0^x k(x,y)f(y)\,dy,\] where
$\alpha$ is constant and $k : [0, r]^2 \to\mathbb{R}$ is a continuous
function. Prove that if
\[ |\alpha|+\sup_{x\in [ 0, r]}\int_0^x |k(x,y)|\,dy = \lambda < 1,\]
then $T$ is a contraction.

\textbf{Proof.} We have, for each $x$, \[\begin{aligned}
|T(f)(x)-T(g)(x)| &\le \left| \alpha (f(x) - g(x)) + \int_0^x k(x,y) (f(y)-g(y))\, dy\right| \\
&\le \left| \alpha (f(x) - g(x)) \right|+ \int_0^x |k(x,y)|| f(y)-g(y)|\, dy \\
&\le \alpha \|f-g\| +\|f-g\| \int_0^x|k(x,y)|\,dy \\
&\le \|f - g\|\left (|\alpha| + \sup_{x\in [0,r]} \int_0^x |k(x,y)|\,dy \right)\\
&= \lambda \|f-g\|.
\end{aligned}\]

This is true for all $x\in [0,r]$, so
$\|T(f) - T(g)\| \le \lambda \|f -g\|$. By $\lambda < 1$, we conclude
that $T$ is a contraction. $\Box$


    % Add a bibliography block to the postdoc
    
    
    
    \end{document}
