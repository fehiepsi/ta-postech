
% Default to the notebook output style

    


% Inherit from the specified cell style.




    
\documentclass{article}

    
    
    \usepackage{graphicx} % Used to insert images
    \usepackage{adjustbox} % Used to constrain images to a maximum size 
    \usepackage{color} % Allow colors to be defined
    \usepackage{enumerate} % Needed for markdown enumerations to work
    \usepackage{geometry} % Used to adjust the document margins
    \usepackage{amsmath} % Equations
    \usepackage{amssymb} % Equations
    \usepackage[mathletters]{ucs} % Extended unicode (utf-8) support
    \usepackage[utf8x]{inputenc} % Allow utf-8 characters in the tex document
    \usepackage{fancyvrb} % verbatim replacement that allows latex
    \usepackage{grffile} % extends the file name processing of package graphics 
                         % to support a larger range 
    % The hyperref package gives us a pdf with properly built
    % internal navigation ('pdf bookmarks' for the table of contents,
    % internal cross-reference links, web links for URLs, etc.)
    \usepackage{hyperref}
    \usepackage{longtable} % longtable support required by pandoc >1.10
    

    

    
    % Prevent overflowing lines due to hard-to-break entities
    \sloppy 
    % Setup hyperref package
    \hypersetup{
      breaklinks=true,  % so long urls are correctly broken across lines
      colorlinks=true,
      urlcolor=blue,
      linkcolor=darkorange,
      citecolor=darkgreen,
      }
    % Slightly bigger margins than the latex defaults
    
    \geometry{verbose,tmargin=1in,bmargin=1in,lmargin=1in,rmargin=1in}
    
    

    \begin{document}
    
    
    
    
    

    
    \subsection*{Analysis I - homework - week
4}\label{analysis-i---homework---week-4}
\addcontentsline{toc}{subsection}{Analysis I - homework - week 4}

    \textbf{1.} Prove the following ``glueing lemma'': Let
$f:[a,b]\to \mathbb{R}^m$ and $g:[b,c]\to \mathbb{R}^m$ be continuous.
Define $h:[a,c]\to\mathbb{R}^m$ by $h=f$ on $[a,b)$ and $h=g$ on
$[b,c]$. If $f(b) = g(b)$, then $h$ is continuous.

\textbf{Proof.} Let $x \in [a,c]$ and a sequence $\{x_n\}\subset [a,c]$
converging to $x$. If $x\in [a,b)$, then there is $\delta >0$ such that
$D(x,\delta)\cap [a,c] \subset [a,b)$. Moreover, there is $N > 0$ such
that $x_n \in D(x,\delta)$ for all $n\ge N$. Because $\{x_n\}_{n\ge N}$
is a sequence in $[a,b)$ and converges to $x\in [a,b)$, we have
$\{f(x_n)\}_{n\ge N}$ converges to $f(x)$. By $h(x_n) = f(x_n)$ for all
$n\ge N$, we have $\{h(x_n)\}_{n\ge N}$ converges to $h(x)=f(x)$. This
means that $h(x_n)$ converges to $h(x)$ as $n \to \infty$.

Similarly, if $x\in (b,c]$, we have $h(x_n)$ converges to $h(x)$ as
$n\to \infty$. The remaining case is when $x = b$. In this case,
$h(x) = f(x) = g(x)$. Let $\epsilon >0$. By the continuity of $f$ at
$b$, there is $\delta_1 > 0$ such that $|f(y)-f(x)|\le \epsilon$ if
$y\in D(x,\delta_1)\cap [a,b]$. Similarly, there is $\delta_2 > 0$ such
that $|g(y)-g(x)|\le \epsilon$ if $y\in D(x,\delta_2)\cap [b,c]$. Let
$\delta = \min\{\delta_1,\delta_2\}$. If $y\in D(x,\delta)\cap [a,b)$,
then $h(y) = f(y)$ and $|h(y)-h(x)| = |f(y)-f(x)| \le \epsilon$. If
$y\in D(x,\delta)\cap [b,c]$, then $h(y) = g(y)$ and
$|h(y)-h(x)| = |g(y)-g(x)| \le \epsilon$. So if
$y\in D(x,\delta)\cap [a,c]$, then $|h(y)-h(x)| \le \epsilon$. The
existence of $\delta$ is true for all $\epsilon > 0$, so $h$ is
continuous at $x=b$. We conclude. $\Box$

\emph{Note.} This proof gives applications of both $\epsilon-\delta$
argument and ``converging sequence'' argument when proving a function is
continuous. Just use the one we feel more comfortable in each situation.

    \textbf{2.} $(a)$ For $f:(a,b)\to \mathbb{R}$, show that if $f$ is
continuous then its graph $\Gamma$ is path-connected and vice versa.

$(b)$ For $f:A\to \mathbb{R}^m$, $A\subset \mathbb{R}^n$, show that for
$n\ge 2$, connectedness of the graph does not imply continuity. {[}Hint:
For $f:\mathbb{R}^2 \to \mathbb{R}$, cut a slit in the graph.{]}

$(c)$ Discuss $(b)$ for $m=n=1$.

\textbf{Proof.} $(a)$ For the $(\Rightarrow)$ side, let $(x,f(x))$ and
$(y,f(y))$ be two distinct points in $\Gamma$. The mapping $\varphi$ is
defined from $[0,1]$ to $\Gamma$ by
$\varphi(t) = ((1-t)x + ty,f((1-t)x+ty))$. By the continuity of $f$, we
have $\varphi$ is continuous on $[0,1]$. Moreover,
$\varphi(0) = (x,f(x))$ and $\varphi(1) = (y,f(y))$. So $(x,f(x))$ is
path-connected to $(y,f(y))$ in $\Gamma$. We conclude that $\Gamma$ is
path-connected.

For the $(\Leftarrow)$ side, suppose $\Gamma$ is path-connected. Let
$x\in (a,b)$, we will show that $f$ is continuous at $x$. Indeed, let
$y, z \in (a,b)$ such that $y < x < z$ and let $\pi$ be the project of
$\Gamma$ which is defined by $\pi(x,y) = x$. It is clear that $\pi$ is
continuous. Because $\Gamma$ is path-connected, there exists a
continuous function $\varphi:[0,1]\to \Gamma$ such that
$\varphi(0) = (y,f(y))$ and $\varphi(1) = (z,f(z))$. We have
$\pi\circ \varphi$ is continuous, so $\pi\circ \varphi([0,1])$ must be
connected, which implies that $[y,z]\subset \pi\circ \varphi([0,1])$,
hence $x$ is a interior point of $\pi\circ \varphi([0,1])$.

We have $\varphi([0,1])$ is a compact set. Now we consider $\pi$ is
restricted to $\varphi([0,1])$, then $\pi$ is continuous, $1-1$, and
onto $\pi(\varphi([0,1]))$. Apply Exercise 5.a, we get $\pi^{-1}$ is a
continuous function. Note that $\pi^{-1}(w) = (w,f(w))$. Because $x$ is
a interior point of $\pi\circ \varphi([0,1])$, and $\pi^{-1}$ is
continuous on $\pi\circ \varphi([0,1])$, $f$ must be continuous at $x$.
We conclude.

$(b)$ Let $g:[0,\infty)\times \{0\} \to \mathbb{R}$ be defined by
$g(x,0) = \sin(1/x)$ if $x > 0$ and $g(0,0) = 0$. By $(c)$ and the
continuity of the injection $(x,f(x)) \mapsto (x,0,g(x,0))$ from the
graph of $f$ in $(c)$ to the graph of $g$, the graph of $g$ must be
connected. But $g$ is discontinuous at $(0,0)$ by a similar reason as
$f$ in $(c)$.

$(c)$ Let $f:[0,\infty)\to \mathbb{R}$ be defined by $f(x) = \sin(1/x)$
if $x > 0$ and $f(0) = 0$. Because $f(1/\pi n) = 0$ for all $n$ and
$(1/\pi n, 0) \to (0,0)$ as $n\to\infty$, we have $(0,0)$ is in the
closure of $\{(x,f(x))\mid x > 0\}$. Note that the set
$\{(x,f(x))\mid x > 0\}$ is connected is a consequence of $(a)$. So we
have the graph of $f$ is connected. Finally, it is clear that $f$ is
disconnected at $0$. For example, the sequence $x_n = 1/(\pi n +\pi/2)$
converges to $0$ as $n\to \infty$ but $f(x_n) = 1$ if $n$ is even and
$f(x_n) = 0$ if $n$ is odd. So, in this case, the connectedness of the
graph also does not imply continuity. $\Box$

\emph{Note.} In $(b)$, I don't quite understand the hint, so I use the
result of $(c)$ as a trick.

    \textbf{3.} Is the sum (product) of two Lipschitz continuous functions
again Lipschitz continuous?

\textbf{Proof.} $(sum)$ True. Let $f_1, f_2$ are two Lipschitz
continuous functions on $A$. Then there is $L_1,L_2 \ge 0$ such that
$\|f_1(x)-f_1(y)\|\le L_1\|x-y\|$ and $\|f_2(x)-f_2(y)\|\le L_2\|x-y\|$
for all $x,y\in A$. We have \[\begin{aligned}
\|(f_1+f_2)(x) - (f_1+f_2)(y)\| &= \|(f_1(x)-f_1(y)) + (f_2(x)-f_2(y))\| \\
&\le \|f_1(x)-f_1(y)\| + \|f_2(x)-f_2(y)\|\\
&\le (L_1+L_2)\|x-y\|.
\end{aligned}\]

So $f_1 + f_2$ is Lipschitz continuous.

$(product)$ False. Let
$f_1 \equiv f_2 \equiv f :\mathbb{R}\to \mathbb{R}$ defined by $f(x)=x$.
It is clear that $f$ is Lipschitz continuous with the constant $L = 1$.
We claim that the function $g:\mathbb{R}\to \mathbb{R}$ defined by
$g(x) = x^2$ is not Lipschitz continuous. Indeed, for any $x\ne 0$, we
have \[|g(x)-g(0)|/|x-0| = |x^2|/|x| = |x|.\]

So if $g$ is Lipschitz with the constant $L\ge 0$, then $L \ge |x|$ for
all $x\ne 0$. But this cannot be true because $L$ is finite. We
conclude. $\Box$

    \textbf{4.} $(a)$ Find a function $f:\mathbb{R}^2\to \mathbb{R}$ such
that \[
\lim_{x\to 0}\lim_{y\to 0} f(x,y),\qquad \lim_{y\to 0}\lim_{x\to 0} f(x,y)
\] exists but are not equal.

$(b)$ Find a function $f:\mathbb{R}^2 \to \mathbb{R}$ such that the two
limits in $(a)$ exist and are equal, but $f$ is not continuous.

$(c)$ Find a function $f:\mathbb{R}^2 \to \mathbb{R}$ which is
continuous on every line through the origin but is not continuous.

\textbf{Proof.} $(a)$ Let $f$ be defined by $f(x,y) = x/(x+y)$ with
$(x,y)\ne (0,0)$ and $f(0,0) = 0$. Then for all $x\ne 0$, we have
$\lim_{y\to 0} x/(x+y) = 1$. So
\[ \lim_{x\to 0}\lim_{y\to 0} f(x,y) = 1.\]

However, for all $y\ne 0$, we have $\lim_{x\to 0} x/(x+y) = 0$. So
\[ \lim_{y\to 0}\lim_{x\to 0} f(x,y) = 0 \ne  \lim_{x\to 0}\lim_{y\to 0} f(x,y).\]

$(b)$ Let $f$ be defined by $f(x,y) = xy/(x^2+y^2)$ with
$(x,y)\ne (0,0)$ and $f(0,0) = 0$. Then for all $x\ne 0$, we have
$\lim_{y\to 0} f(x,y) = 0$. Hence
\[ \lim_{x\to 0}\lim_{y\to 0} f(x,y) = 0.\]

Similarly, \[ \lim_{y\to 0}\lim_{x\to 0} f(x,y) = 0.\]

But $f$ is not continuous at $(0,0)$ because $f(1/n,1/n) = 1/2$ for all
$n$.

$(c)$ Let $f$ be defined by $f(x,y) = xy^2/(x^2+y^4)$ with
$(x,y)\ne (0,0)$ and $f(0,0) = 0$. Then on the line $x = 0$, we have
$f(0,y) = 0$ for all $y$. So $f$ is continuous on the line $[x=0]$. For
other lines through origin, the formula for each line is $[y = ax]$ for
some $a\in \mathbb{R}$. One the line $[y=ax]$, for $x\ne 0$, we have
$f(x,ax) = a^2x/(1+ a^4x^2)$, so we just have to worry about the
continuity of $f$ at $(0,0)$. But $a^2x/(1+ a^4x^2) \to 0$ as $x \to 0$.
So $f$ is truly continuous at $(0,0)$ on the line $[y=ax]$. This means
that $f$ is continuous on every line through the origin.

Finally, we claim that $f$ is discontinuous at $(0,0)$. Indeed, for
example, we have $(1/n^2,1/n)\to (0,0)$ but $f(1/n^2,1/n) = 1/2 \to 1/2$
as $n\to \infty$. $\Box$

    \textbf{5.} $(a)$ Suppose $f$ is a continuous $1-1$ mapping of a compact
metric space $X$ onto a metric space $Y$. Then the inverse mapping
$f^{-1}$ is a continuous mapping of $Y$ onto $X$.

$(b)$ Give a counterexample that $f$ is a continuous $1-1$ mapping of a
metric space $X$ onto a metric space $Y$ but the inverse mapping
$f^{-1}$ is a discontinuous mapping of $Y$ onto $X$.

\textbf{Proof.} $(a)$ Let $\{y_n\}$ be a sequence in $Y$ which converges
to a point $y\in Y$. Put $x=f^{-1}(y)$ and $x_n = f^{-1}(y_n)$ for all
$n$. To show $f^{-1}$ is continuous, we just need to prove that
$x_n \to x$ as $n\to \infty$. Suppose on the contrary that $\{x_n\}$
does not converge to $x$. This means that there exists $\epsilon > 0$
such that: ``for all $n$, there exists $m > n$ such that
$d(x_m,x) > \epsilon$''. Then there exists a subsequence $\{x_{n_k}\}$
of $\{x_n\}$ such that $d(x_{n_k},x) > \epsilon$ for all $k$. The
compactness of $X$ gives us a subsequence $\{x_{n_{k_t}}\}$ of
$\{x_{n_k}\}$ which converges to $z\in X$. By
$d(x_{n_{k_t}},x) > \epsilon$ for all $t$, we have
$d(z,x) \ge \epsilon$, hence $z\ne x$. Now, the continuity of $f$
implies that $y_{n_{k_t}}= f(x_{n_{k_t}})$ converges to $f(z)$ as
$t\to \infty$, hence $f(z) = y = f(x)$. This contradicts with $z\ne x$
(because $f$ is $1-1$).

$(b)$ Let $X = [0,1) \cup [2,3]$, $f(x) = x$ on $[0,1)$, and
$f(x) = x-1$ on $[2,3]$. Clear that $f$ is continuous on $[0,1)$ and
$[2,3]$, hence continuous on $X$. Let $Y = f(X) = [0,2]$, we have $f$ is
a $1-1$ mapping from $X$ onto $Y$. The inverse mapping $f^{-1}$ is
defined by $f^{-1}(y) = y$ for $y \in [0,1)$ and $f^{-1}(y) = y+1$ for
$y\in [1,2]$. So $f^{-1}$ is not continuous at $1$. We get a
counterexample. $\Box$

    \textbf{6.} Let $f:(0,1)\to \mathbb{R}$ be a monotone increasing
function. Define $D:=\{x\in (0,1)\mid f\text{ is discontinuous at }x\}$.
Prove that $D$ is at most countable.

\textbf{Proof.} Suppose that $f$ is discontinous at $x$. Because $f$ is
monotone increasing, the limit of $f(y)$ as $y \to x$ from below exists.
We denote this limit by $a_x$. Similarly, we denote $b_x = \lim f(z)$ as
$z\to x$ from above. Because $f$ is discontinuous at $x$, $a_x < b_x$
(if not, $a_x = b_x$, hence $a_x=b_x=f(x)$, hence $f$ is continuous at
$x$). There always exists a rational number in $(a_x,b_x)$. By Axiom of
Choice, there exists a function $q$ on $D$ such that
$q(x)\in \mathbb{Q}$ and $q(x) \in (a_x,b_x)$. For $x,y\in D$ and
$x < y$, we have $q(x) < b(x) \le a(y) < q(y)$. So $q$ is strictly
increasing. In particular, $q$ is an injective function from $D$ to
$\mathbb{Q}$. Hence $D$ is at most countable. $\Box$

    \textbf{7.} Prove that every convex function is continuous.

\textbf{Proof.} Let $f$ is a convex function in $(a,b)$. Let
$x\in (a,b)$. It is enough to show that $f$ is right continuous at $x$
(the left continuity is similar). Let $y, z \in (a,b)$ such that
$z < x < y$ and a sequence $\{x_n\}\subset (x,y)$ which converges to
$x$. Let $\lambda_n,\alpha_n\in (0,1)$ be such that
$x_n = \lambda_n x + (1-\lambda_n)y$ and
$x = \alpha_n z + (1-\alpha_n)x_n$. We have $\lambda_n = (y-x_n)/(y-x)$,
hence converges to $1$ as $n\to \infty$; and
$\alpha_n = (x_n-x)/(x_n-z)$, hence converges to $0$ as $n\to \infty$.
By the convexity of $f$, we have \[
f(x_n) \le \lambda_n f(x) + (1-\lambda_n)f(y)
\] and \[
f(x) \le \alpha_n f(z) + (1-\alpha_n) f(x_n).
\]

The second inequality is equivalent to \[
\frac{f(x) - \alpha_n f(z)}{1-\alpha_n} \le f(x_n).
\]

So we have \[
\frac{f(x) - \alpha_n f(z)}{1-\alpha_n} \le f(x_n) \le \lambda_n f(x) + (1-\lambda_n)f(y).
\]

Both the left sequence and the right sequence converges to $f(x)$ as
$n \to\infty$. So $f(x_n) \to f(x)$ as $n\to \infty$. We conclude.
$\Box$

    \textbf{8.} Let $f:\mathbb{R}\to \mathbb{R}$.

$(a)$ Give an example that $f$ is discontinuous at every point in
$\mathbb{R}$.

$(b)$ Give an example that $f$ is continuous at every irrational point,
and that $f$ is discontinuous at every rational point.

\textbf{Proof.} $(a)$ Consider the function $f \equiv \chi_{\mathbb{Q}}$
with value $1$ on $\mathbb{Q}$ and $0$ on the remaining. Let
$x\in \mathbb{R}$. If $x\in \mathbb{Q}$, we consider the sequence
$\{x_n\}$ of irrational numbers which converging to $x$. Then by
$f(x_n) = 0$ for all $n$ and $f(x) = 1$, we cannot have $f$ continous at
$x$. Similarly, we also get $f$ is not continous at $x$ when
$x\notin \mathbb{Q}$. We conclude.

$(b)$ Let $f$ be the non-negative function defined by $f(x) = 0$ if $x$
is irrational and $f(x) = 1/q_x$ if $x$ is rational, here $q_x$ is the
least positive integer such that $q_xx \in \mathbb{Z}$ (in other word,
$q_x$ is the denominator of $x$ in its lowest terms $p/q$, $q>0$). Note
that $f(0) = 1$. First, we claim that $f$ is discontinuous at every
rational point. Let $x\in \mathbb{Q}$, then $f(x) \ne 0$. But we can
construct a sequence of irrational point $\{x_n\}$ which converges to
$x$. By $f(x_n) = 0$ for all $n$, we conclude that $f$ is not continuous
at $x$.

Now, we prove that $f$ is continuous at every irrational point. Let
$x\notin \mathbb{Q}$ and $\epsilon > 0$. Pick $N\in \mathbb{N}$ such
that $1/N < \epsilon$. By $f(x) = 0$, to show that $f$ is continuous at
$x$, it is enough to give $\delta > 0$ such that $f(y) < 1/N$ for all
$y \in (x-\delta, x+\delta)$. We claim that for each $i\in \mathbb{N}$,
the number of rational number $z\in (x-1,x+1)$ with $q_z = i$ is finite.
Indeed, by $q_zz \in \mathbb{Z}$, we have $iz \in \mathbb{Z}$, but the
number of integers in $(i(x-1),i(x+1))$ is finite. So the number of such
$z$ must be finite. We denote the set of these $z$ by $A_i$, for each
$i$.

Let $A$ be the union of $A_i$ with $1\le i\le N$. Because $A$ is finite,
there is $\delta > 0$ such that $|x-z| > \delta$ for all $z\in A$. This
is the $\delta$ we want to find. Indeed, for
$y\in (x-\delta, x+\delta)$, $y \notin A$ by the way we choose $\delta$.
If $y\notin \mathbb{Q}$, clearly $f(y) = 0 < 1/N$. If $y\in \mathbb{Q}$,
by $y\notin A$, we have $q_y  > N$, hence $f(y) = 1/q_y < 1/N$, as
required. $\Box$

\emph{Note.} The function in $(b)$ is called Thomae's function (or
popcorn function, or raindrop function, \ldots{})

    \textbf{9.} Show that ($A$ is connected and locally path-connected)
$\Leftrightarrow$ ($A$ is path-connected).

\textbf{Proof.} For the $(\Rightarrow)$ side, let $x\in A$ and let $V$
be all points in $A$ which can be connected to $x$ by a path in $A$.
Because $x\in V$, $V\ne \varnothing$. For a point $y\in V$, by
assumption, there exists a neighborhood $U_y$ such that $A\cap U_y$ is
path-connected. Hence every point in $A \cap U_y$ can be connected to
$y$ by a path in $A$ (indeed, in $A\cap U_y$). So every point in
$A\cap U_y$ can be connected to $x$ by a path in $A$. So
$(A\cap U_y) \subset V$, which implies that $V$ is open relative to $A$.

Now, let $y$ in the closure of $V$ in $A$. As above, there eixsts a
neighborhood $U_y$ such that $A\cap U_y$ is path-connected. Because
$A\cap U_y$ is open relative to $A$, we have
$(A\cap U_y)\cap  V \neq \varnothing$. Let $z\in (A\cap U_y)\cap  V$,
then $z$ can be connected by a path to $x$ and a path to $y$. So $y$ can
be connected to $x$ by a path in $A$, which means $y \in V$. Hence $V$
is closed relative to $A$.

In conclusion, $V$ and $A\backslash V$ are disjoint and open relative to
$A$. By the connectedness of $A$, $A\backslash V$ must be $\varnothing$.
Hence $V = A$, which means $A$ is path-connected.

For the $(\Leftarrow)$ side, just let $U$ be the universal set which we
are working with, we get $A\cap U$ ($=A$) is path-connected. Hence $A$
is locally path-connected. By Theorem 3.3, the path-connectedness of $A$
implies it connected. $\Box$


    % Add a bibliography block to the postdoc
    
    
    
    \end{document}
