
% Default to the notebook output style

    


% Inherit from the specified cell style.




    
\documentclass{article}

    
    
    \usepackage{graphicx} % Used to insert images
    \usepackage{adjustbox} % Used to constrain images to a maximum size 
    \usepackage{color} % Allow colors to be defined
    \usepackage{enumerate} % Needed for markdown enumerations to work
    \usepackage{geometry} % Used to adjust the document margins
    \usepackage{amsmath} % Equations
    \usepackage{amssymb} % Equations
    \usepackage[mathletters]{ucs} % Extended unicode (utf-8) support
    \usepackage[utf8x]{inputenc} % Allow utf-8 characters in the tex document
    \usepackage{fancyvrb} % verbatim replacement that allows latex
    \usepackage{grffile} % extends the file name processing of package graphics 
                         % to support a larger range 
    % The hyperref package gives us a pdf with properly built
    % internal navigation ('pdf bookmarks' for the table of contents,
    % internal cross-reference links, web links for URLs, etc.)
    \usepackage{hyperref}
    \usepackage{longtable} % longtable support required by pandoc >1.10
    \usepackage{booktabs}  % table support for pandoc > 1.12.2
    

    

    
    % Prevent overflowing lines due to hard-to-break entities
    \sloppy 
    % Setup hyperref package
    \hypersetup{
      breaklinks=true,  % so long urls are correctly broken across lines
      colorlinks=true,
      urlcolor=blue,
      linkcolor=darkorange,
      citecolor=darkgreen,
      }
    % Slightly bigger margins than the latex defaults
    
    \geometry{verbose,tmargin=1in,bmargin=1in,lmargin=1in,rmargin=1in}
    
    

    \begin{document}
    
    
    
    
    

    
    \subsection*{Analysis I - homework - week
12}\label{analysis-i---homework---week-12}
\addcontentsline{toc}{subsection}{Analysis I - homework - week 12}

    \textbf{1.} Let $f:A\subset\mathbb{R}\to \mathbb{R}$, where $A$ is
bounded and $f$ is bounded and integrable over $A$. Consider another
bounded integrable function $g:A\to \mathbb{R}$ such that $g(x) = f(x)$
except on a set $S\subset A$ of measure zero. Then assuming $f$ and $g$
are integrable on $S$ and $A\backslash S$, prove $\int_A g = \int_A f$.

    \textbf{Sketch of proof.} Apply Theorem 8.4 and 8.5. $\Box$

    \textbf{2.} Prove that an increasing function $f:[a,b]\to \mathbb{R}$ is
Riemann integrable.

    \textbf{Sketch of proof.} For each interval $[c,d] \subset [a,b]$, we
have $\sup{f([c,d])} - \inf{f([c,d])} \le f(b) - f(a)$. So for any
partition $P$ of $[a,b]$, we have
$0\le U(f,P) - L(f,P) \le (f(b) - f(a)) |P|$, where $|P|$ denotes the
maximal length of intervals in the partition $P$. From this, show that
$f$ is Riemann integrable. $\Box$

    \textbf{3.} If $f:A\subset \mathbb{R}^n \to \mathbb{R}$ and
$g:A \subset \mathbb{R}^n \to \mathbb{R}$ are bounded integrable
functions on the bounded set $A$ such that $f(x) < g(x)$ for all
$x\in A$ and $v(A) \ne 0$, then show $\int_A f < \int_A g$.

    \textbf{Sketch of proof.} Suppose on the contrary that
$\int_A f = \int_A g$. Show that $A$ has measure zero. From $A$ has
measure zero and $A$ has volume, we cannot immediately conclude that
$v(A) = 0$ to get a contradiction. We are going to prove this.

By $A$ is bounded, we have $\bar{A}$ is compact. By $A$ is bounded and
$A$ has volume, $\partial A$ has measure zero. So $\bar{A}$ has measure
zero. Let $\epsilon > 0$, there exists a covering of $\bar{A}$ by a
countable number of open rectangles such that
$\sum_{i=1}^{\infty} v(S_i) < \epsilon$ (see the note at the end of the
proof of Theorem 8.1 in textbook). The compactness of $\bar{A}$ implies
that there exists a finite number of rectangles
$S_{i_1}, \ldots,S_{i_n}$ which cover $A$. Conclude that $v(A) = 0$ to
get a contradiction. $\Box$

    \textbf{4.} $(a)$ Show that a bounded set $A \subset \mathbb{R}^n$ has
zero volume if and only if it can be covered by a finite number of
rectangles of arbitrarily small total volume.

$(b)$ Show that $A\subset \mathbb{R}^n$ has zero volume then it can be
covered by a countable rectangles of arbitrarily small total volume.

    \textbf{Sketch of proof.} $(a)$ See Worked Example 1 for Chapter 8.

$(b)$ Let $1_A$ be the indicator function of $A$ in $\mathbb{R}^n$. Show
that the set at which $1_A$ discontinuous is $\partial A$.

By Theorem 8.3 (with valid for $A$ unbounded), if $A$ has zero volume,
then $1_A$ is integrable, so $\partial A$ has measure zero. Moreover,
$v(A) = 0$ also implies $\operatorname{int}(A) = \varnothing$ (because
each disc contains a nontrivial rectangle, hence has a positive volume).
So $A \subset \partial A$, which implies $A$ has measure zero. $\Box$

    \textbf{5.} Let $B\subset \mathbb{R}^n$ be bounded, $f:B\to \mathbb{R}$
be integrable, $f\ge 0$. If $A\subset B$ and $f$ is integrable on $A$,
then $\int_A f \le \int_B f$. Is this true if we do not assume
$f \ge 0$?

    \textbf{Sketch of proof.} Let $R$ be a rectangle which encloses $B$,
then we have $\int_A f = \int_R 1_A f$ and $\int_B f = \int_R 1_B f$.
Show that $\int_A f \le \int_B f$. If we do not assume $f \ge 0$, this
is not true in general. For example, let $B = [0,1]$, $f = -1$ on $B$,
and $A = \varnothing$. $\Box$

    \textbf{6.} Let $A\subset \mathbb{R}^n$ be bounded. Prove that if
$f :A\subset \mathbb{R}^n \to \mathbb{R}$ is continuous, $A$ is open
with volume, and $\int_B f = 0$ for each $B\subset A$ with volume, then
$f = 0$.

    \textbf{Sketch of proof.} Suppose on the contrary that there exists
$x \in A$ such that $f(x) \ne 0$. Without loss of generality, suppose
more that $f(x) = a > 0$. Then by the continuity of $f$, there exists a
disk $D\equiv D(x,\delta) \subset A$ such that $f > a/2$ on $D$. Show
that $\int_D f > 0$ to get a contradiction. $\Box$

    \textbf{7.} Show that the Cantor set $C \subset [0,1]$ has measure zero.

    \textbf{Sketch of proof.} Let $F_1 = [0,1/3] \cup [2/3,1]$ be obtained
from $[0,1]$ by removing the middle third. Repeatly, obtaining
\[F_2 = [0,1/9]\cup [2/9,3/9] \cup [6/9,7/9] \cup [8/9,1].\]

In general, $F_n$ is a union of closed intervals and $F_{n+1}$ is
obtained from $F_n$ by removing the middle third of these intervals.
Then by definition, $C = \bigcap_{1}^{\infty} F_n$.

For each $n$, we have $v(F_n) = (2/3)^n$. So for $\epsilon > 0$, there
exists $n$ large enough such that $v(F_n) < \epsilon /2$. Show that
there exists a finite covering of $F_n$ by rectangles of total volume
less than $\epsilon$. This covering of $F_n$ also encloses $C$ and has
total volume less than $\epsilon$. So $C$ has volume zero, in
particular, $C$ has measure zero. $\Box$

    \textbf{8.} Let $f:[0,1]\to \mathbb{R}$ be a bounded and $f^2$ is a
Riemann integrable function. Does it follow that $f$ is a Riemann
integrable function too?

    \textbf{Sketch of proof.} No. Let $f$ be defined by $f(x) = 1$ if
$x \in \mathbb{Q}$ and $f(x) = -1$ otherwise. $\Box$

    \textbf{9.} Let $p$ and $q$ be positive real numbers such that
$p^{-1}+q^{-1} =1$. Prove the following statements.

$(a)$ If $u\ge 0$ and $v\ge 0$, then
\[uv \le \frac{u^p}{p} + \frac{v^q}{q}.\]

Equality holds if and only if $u^p = v^q$.

$(b)$ Let $f:A\subset \mathbb{R}^n \to \mathbb{R}$ and
$g:A\subset \mathbb{R}^n \to \mathbb{R}$ be non-negative integrable
functions. Prove that if \[\int_A f^p = \int_A g^q = 1,\] then
\[\int_A fg \le 1.\]

$(c)$ Prove that if $f:A \subset \mathbb{R}^n \to \mathbb{R}$ and
$g:A\subset \mathbb{R}^n \to \mathbb{R}$ are integrable functions, then
\[\left|\int_A fg\right| \le \left(\int_A |f|^p\right)^{1/p} \left(\int_A |g|^q\right)^{1/q}.\]

    \textbf{Sketch of proof.} See Problem 6 of HW\#7. $\Box$

    \textbf{10.} Let $f,g,h:A\subset \mathbb{R}^n \to \mathbb{R}$ be
integrable functions. Define
\[\|f\|_2 := \left(\int_A|f|^2\right)^{1/2}.\]

Prove the following inequality,
\[\|f-h\|_2 \le \|f-g\|_2 + \|g- h\|_2.\]

    \textbf{Sketch of proof.} It is enough to show that if
$u,v :A \to \mathbb{R}$ be integrable functions, then
$\|u + v\|_2 \le \|u\|_2 + \|v\|_2$. Indeed, we have
\[\|u+v\|_2^2 \le \int_A |u+v| (|u|+|v|)= \int_A|u+v||u| + \int_A |u+v||v|  \le  \|u+v\|_2(\|u\|_2 + \|v\|_2).\Box\]


    % Add a bibliography block to the postdoc
    
    
    
    \end{document}
