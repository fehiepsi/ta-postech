
% Default to the notebook output style

    


% Inherit from the specified cell style.




    
\documentclass{article}

    
    
    \usepackage{graphicx} % Used to insert images
    \usepackage{adjustbox} % Used to constrain images to a maximum size 
    \usepackage{color} % Allow colors to be defined
    \usepackage{enumerate} % Needed for markdown enumerations to work
    \usepackage{geometry} % Used to adjust the document margins
    \usepackage{amsmath} % Equations
    \usepackage{amssymb} % Equations
    \usepackage[mathletters]{ucs} % Extended unicode (utf-8) support
    \usepackage[utf8x]{inputenc} % Allow utf-8 characters in the tex document
    \usepackage{fancyvrb} % verbatim replacement that allows latex
    \usepackage{grffile} % extends the file name processing of package graphics 
                         % to support a larger range 
    % The hyperref package gives us a pdf with properly built
    % internal navigation ('pdf bookmarks' for the table of contents,
    % internal cross-reference links, web links for URLs, etc.)
    \usepackage{hyperref}
    \usepackage{longtable} % longtable support required by pandoc >1.10
    

    

    
    % Prevent overflowing lines due to hard-to-break entities
    \sloppy 
    % Setup hyperref package
    \hypersetup{
      breaklinks=true,  % so long urls are correctly broken across lines
      colorlinks=true,
      urlcolor=blue,
      linkcolor=darkorange,
      citecolor=darkgreen,
      }
    % Slightly bigger margins than the latex defaults
    
    \geometry{verbose,tmargin=1in,bmargin=1in,lmargin=1in,rmargin=1in}
    
    

    \begin{document}
    
    
    
    
    

    
    \subsection*{Analysis I - homework - week
9}\label{analysis-i---homework---week-9}
\addcontentsline{toc}{subsection}{Analysis I - homework - week 9}

    \textbf{1.} Let $f(x,y)$ be a real-valued function on $\mathbb{R}^2$.
Use the proof of Theorem 6.9 to show that if $f$ is of class $C^1$ and
$\partial^2 f/\partial x \partial y$ exists and is continuous, then
$\partial^2f /\partial y \partial x$ exists, and
\[\frac{\partial^2 f}{\partial x \partial y} = \frac{\partial^2f }{\partial y \partial x}.\]

    \textbf{Sketch of proof.} Put
\[S_{h,k} = f(x+h,y+k) + f(x,y) - f(x+h,y) - f(x,y+k).\]

By Theorem 6.9, we have
\[S_{h,k} = \frac{\partial^2 f}{\partial x \partial y}(c_{h,k},d_{h,k}).hk=\left( \frac{\partial f}{\partial x}(e_{h,k}, y + k) - \frac{\partial f}{\partial x}(e_{h,k}, y )\right).h \]
for some $c_{h,k},e_{h,k}$ between $x$ and $x+h$, $d_{h_k}$ between $y$
and $y+k$. So we have
\[ \frac{\partial^2 f}{\partial x \partial y}(c_{h,k},d_{h,k})=\left( \frac{\partial f}{\partial x}(e_{h,k}, y + k) - \frac{\partial f}{\partial x}(e_{h,k}, y )\right)/k. \]

Let $h \to 0$, then let $k\to 0$. Conclude. $\Box$

    \textbf{2.} Let $f:A\to \mathbb{R}$ be continuous,
$A\subset \mathbb{R}^n$ open. Assume that all directional derivatives
exist and define, at each $x_0\in A$, a linear map $Df(x_0)$. Must $f$
be differentiable?

    \textbf{Sketch of proof.} No. For example, let
$f : \mathbb{R}^2 \to \mathbb{R}$ be defined by $f(0,0) = 0$ and
$f(x,y) = x^2y\sqrt{x^2+y^2}/(x^4+y^2)$ if $(x,y)\ne (0,0)$. Show that
at $(0,0)$, all directional derivatives exist and be zero, hence define
the linear map $Df(0,0) (x,y) = 0$. But $f$ is not differentiable at
$(0,0)$. Indeed, we have
\[\lim_{(h,k)\to (0,0)}\frac{|f(h,k) - f(0,0) - Df(0,0)(h,k)|}{\|(h,k)\|} = \lim_{(h,k)\to (0,0)} \frac{h^2k}{h^4+ k^2},\]
which does not exists. $\Box$

    \textbf{3.} Suppose $f$ and $g$ are real differentiable in $(a,b)$, and
$g'(x) \ne 0$ for all $x\in (a,b)$. Suppose
\[\lim_{x\to a}\frac{f'(x)}{g'(x)} = A.\] Prove that if
\[\tag{*}\lim_{x\to a} f(x) = \lim_{x\to a}g(x) = 0\] or if
\[\tag{**}\lim_{x\to a} g(x) \to +\infty\] then
\[\lim_{x\to a} \frac{f(x)}{g(x)} = A.\]

    \textbf{Sketch of proof.} Suppose we have $(*)$, first show that for any
$c\in (a,b)$, there exists $d_c \in (a,c)$ such that
\[\frac{f'(d_c)}{g'(d_c)} = \frac{f(c) - f(a)}{g(c) - g(a)}.\]

Then end proof for $(*)$ by showing that
\[\lim_{x \to a} \frac{f(x)}{g(x)} = \lim_{x \to a^+} \left(\frac{f(x) - f(a)}{x-a}\cdot \frac{x-a}{f(x) - f(a)} \right)= \lim_{x\to a^+}\frac{f'(d_x)}{g'(d_x)} = A.\]

Now, we suppose $(**)$. For $0<\epsilon < 1$, there exists $c \in (a,b)$
such that $|f'(x)/g'(x) - A| < \epsilon/2$ for all $x\in (a,c)$. By
$(**)$, we may suppose that $g(x) > 0$ for all $x\in (a,c)$. Now, for
each $x \in (a,c)$, as above, there exists $d_x \in (x,c)$ such that
\[\frac{f'(d_x)}{g'(d_x)} = \frac{f(c) - f(x)}{g(c) - g(x)}.\]

We have \[\begin{aligned}
\frac{f(x)}{g(x)} &= \frac{f(c) - f(x)}{g(c) - g(x)}\left(1 - \frac{g(c)}{g(x) }\right) + \frac{f(c)}{g(x)}\\
&=\frac{f'(d_x)}{g'(d_x)}\left(1 - \frac{g(c)}{g(x) }\right) + \frac{f(c)}{g(x)} .
\end{aligned}\]

Hence \[\begin{aligned}
\left|\frac{f(x)}{g(x)} - \frac{f'(d_x)}{g'(d_x)}\right|&= \left|-\frac{f'(d_x)}{g'(d_x)}\cdot \frac{g(c)}{g(x)} + \frac{f(c)}{g(x)}\right|\\
&\le \frac{ (|A| + 1)|g(c)|+ |f(c)|}{|g(x)|}.
\end{aligned}\]

By $g(x) \to +\infty$ as $x \to a$, there exists $e \in (a,b)$ such that
\[ \frac{ (|A| + 1)|g(c)|+ |f(c)|}{|g(x)|} < \epsilon /2\] for all
$x \in (a,e)$. Put $m = \min \{c,e\}$, then conclude. $\Box$

    \textbf{4.} Let $f:(-1,1)\to \mathbb{R}$ is continuous function and
$f'(x)$ exists for all $x\in (-1,1)\backslash \{0\}$. Prove that if
$\lim_{x\to 0^+} f'(x)$ and $\lim_{x\to 0^-} f'(x)$ exist and
\[\lim_{x\to 0^+} f'(x) = \lim_{x\to 0^-} f'(x),\] then $f'(0)$ exists.

    \textbf{Sketch of proof.} We have
$f'(0) = \lim_{h\to 0}(f(h)-f(0))/h = \lim_{h\to 0} f'(c_h)$ for some
$c_h$ between $0$ and $h$ (by Mean Value Theorem). $\Box$

    \textbf{5.} Let $f$ be defined for all real $x$, and suppose that
\[|f(x) - f(y)| \le (x-y)^2\] for all real $x$ and $y$. Prove that $f$
is constant.

    \textbf{Sketch of proof.} First use Mean Value Theorem to show that $f'$
is $0$. Then conclude $f$ is constant. $\Box$

    \textbf{6.} If
\[C_0 + \frac{C_1}{2} + \cdots + \frac{C_{n-1}}{n} + \frac{C_n}{n+1} = 0,\]
where $C_0,\ldots,C_n$ are real constants, prove that the equation
\[C_0 + C_1x + \cdots + C_{n-1} x^{n-1} + C_nx^n = 0\] has at least one
real root between $0$ and $1$.

    \textbf{Sketch of proof.} Put $f(x) = C_0 + C_1x + \cdots +  C_nx^n$ and
$g(x) = C_0 x + C_1 x^2/2 + \cdots + C_nx^{n+1}/(n+1)$. Use Mean Value
Theorem. $\Box$

    \textbf{7.} Let $f$ be a differentiable real function defined in
$(a,b)$. Prove that $f$ is convex if and only if $f'$ is monotonically
increasing.

    \textbf{Sketch of proof.} For the $(\Rightarrow)$ side, let
$x< y \in (a,b)$. Let $h > 0$ such that $x + h < y < y+h < b$. We have
\[\frac{f(x+h) - f(x)}{h} \le \frac{f(y)-f(x+h)}{y -x-h} \le \frac{f(y+h)-f(y)}{h}.\]

Let $h \to 0$, we get $f'(x) \le f'(y)$. So $f'$ is monotonically
increasing.

For the $(\Leftarrow)$ side, note that for $a < x < y < b$ and
$\lambda \in (0,1)$, we have
\[f(\lambda x + (1-\lambda)y) \le \lambda f(x) + (1-\lambda) f(y) \iff \frac{f(\lambda x + (1-\lambda)y) - f(x)}{1-\lambda} \le \frac{f(y) -f(\lambda x + (1-\lambda)y) }{\lambda}.\]

Use Mean Value Theorem, and the assumption that $f'$ is monotonically
increasing, we conclude that
$f(\lambda x + (1-\lambda)y) \le \lambda f(x) + (1-\lambda) f(y)$.
$\Box$

    \textbf{8.} Let $L:\mathbb{R}^n \to \mathbb{R}^n$ be a linear
isomorphism, and $f(x) = L(x) + g(x)$, where $\|g(x)\| \le M \|x\|^2$
and $f$ is $C^1$. Show $f$ is locally invertible near $0$.

    \textbf{Sketch of proof.} By $L$ is a linear isomorphism, we have
$\det (L) \ne 0$, the Jacobian determinant of $L$ at $0$ is
$JL(0) = \det(DL(0)) = \det(L) \ne 0$.

Now, by $\|g(x)\| \le M\|x\|^2$, we have $g(0) = 0$, then show that
$Dg(0) = 0$. So we have
\[Jf(0) = \det(Df(0)) = \det(DL(0) + Dg(0)) = \det(DL(0)) \ne 0.\]

By Inverse Function Theorem, we conclude $f$ is locally invertible near
$0$. $\Box$

    \textbf{9.} Show that the system of equations \[\begin{aligned}
3x+y -z + u^2 &= 0\\
x-y+2z+u &= 0\\
2x+2y - z + u^2 &= 0\\
\end{aligned}\] can be solved for $x,y,u$ in terms of $z$; but not for
$x,y,z$ in terms of $u$.

    \textbf{Sketch of proof.} The latter claim is wrong. Indeed, the matrix
\[\begin{pmatrix}
3 & 1 & -1\\
1 & -1 & 2 \\
2 & 2 & -1
\end{pmatrix}\] has determinant $-8$, so the system can be solved for
$x,y,z$ in terms of $u$.

Now we solve $x,y,u$ in terms of $z$. This is quite elementary. $\Box$

    \textbf{10.} Let $f(x) = x+2x^2\sin(1/x)$, $x\ne 0$, $f(0) = 0$. Show
that $f'(0) = 1$, $f'$ is bounded in $(-1,1)$, but $f$ is not one-to-one
in any neighborhood of $0$.

    \textbf{Sketch of proof.} To show that $f$ is not one-to-one in any
neighborhood of $0$, it is enough to consider neighborhoods having the
form $(-\epsilon,\epsilon)$ with $0 < \epsilon < 1/2$. If $f$ is
one-to-one in $(0,\epsilon)$, then by $f(0) = 0$ and
$f(\epsilon) = \epsilon(1 + 2\epsilon \sin(1/\epsilon)) > 0$, $f$ must
be nondecreasing on $(0, \epsilon)$. So $f'(x) \ge 0$ for all
$x\in (0,\epsilon)$. Put $x_n =1/( \pi/2 + 2n \pi)$, we have
\[f(x_n) = 1 + \frac{4}{\pi/2 + 2n\pi} -2 < 0\] for all
$n\in \mathbb{N}$. So if we take $n$ large enough such that
$1/( \pi/2 + 2n \pi)< \epsilon$, then $x_n \in (0,\epsilon)$ and
$f'(x_n) < 0$. We get a contradiction. $\Box$


    % Add a bibliography block to the postdoc
    
    
    
    \end{document}
