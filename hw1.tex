
% Default to the notebook output style

    


% Inherit from the specified cell style.




    
\documentclass{article}

    
    
    \usepackage{graphicx} % Used to insert images
    \usepackage{adjustbox} % Used to constrain images to a maximum size 
    \usepackage{color} % Allow colors to be defined
    \usepackage{enumerate} % Needed for markdown enumerations to work
    \usepackage{geometry} % Used to adjust the document margins
    \usepackage{amsmath} % Equations
    \usepackage{amssymb} % Equations
    \usepackage[mathletters]{ucs} % Extended unicode (utf-8) support
    \usepackage[utf8x]{inputenc} % Allow utf-8 characters in the tex document
    \usepackage{fancyvrb} % verbatim replacement that allows latex
    \usepackage{grffile} % extends the file name processing of package graphics 
                         % to support a larger range 
    % The hyperref package gives us a pdf with properly built
    % internal navigation ('pdf bookmarks' for the table of contents,
    % internal cross-reference links, web links for URLs, etc.)
    \usepackage{hyperref}
    \usepackage{longtable} % longtable support required by pandoc >1.10
    

    

    
    % Prevent overflowing lines due to hard-to-break entities
    \sloppy 
    % Setup hyperref package
    \hypersetup{
      breaklinks=true,  % so long urls are correctly broken across lines
      colorlinks=true,
      urlcolor=blue,
      linkcolor=darkorange,
      citecolor=darkgreen,
      }
    % Slightly bigger margins than the latex defaults
    
    \geometry{verbose,tmargin=1in,bmargin=1in,lmargin=1in,rmargin=1in}
    
    

    \begin{document}
    
    
    
    
    

    
    \subsection*{Analysis I - homework - week
1}\label{analysis-i---homework---week-1}
\addcontentsline{toc}{subsection}{Analysis I - homework - week 1}

    \subsubsection*{Section 1.1}\label{section-1.1}
\addcontentsline{toc}{subsubsection}{Section 1.1}

    \textbf{1.} Let $S = \{x | x^3 < 1\}$. Find $\sup(S)$. Is $S$ bounded
below?

\textbf{Proof.} First, we show that $S = \{x \mid x < 1\}$. Indeed,
because $x^2 + x + 1 = (x+\frac{1}{2})^2 + \frac{3}{4} > 0$ for all
$x\in \mathbb{R}$, we have
\[x^3 < 1 \Leftrightarrow (x-1)(x^2 + x + 1) < 0 \Leftrightarrow x - 1 < 0.\]

Now, because $\sup \{x \mid x < 1\} = 1$ and $\{x \mid x<1\}$ is not
bounded below, we have $\sup(S) = 1$ and $S$ is not bounded below.
$\Box$

    \textbf{5.} Let $x_n = \sqrt{n^2+1} - n$. Compute
$\lim_{n\to \infty} x_n$.

\textbf{Proof.} For each $n$, we have
\[x_n = \sqrt{n^2+1}-n = \frac{1}{\sqrt{n^2+1} + n} < \frac{1}{n}.\]

So for $\epsilon > 0$, we can choose $N > \frac{1}{\epsilon}$ and get
\[ 0 < x_n < \frac{1}{n} \le \frac{1}{N} < \epsilon \] for all $n\ge N$.
So $\lim_{n\to\infty}x_n = 0.$ $\Box$

    \subsubsection*{Chapter 1}\label{chapter-1}
\addcontentsline{toc}{subsubsection}{Chapter 1}

    \textbf{10.} For a given sequence $a_n$, we define the numbers
\[ \limsup(a_n) = \inf\{\sup(a_n,a_{n+1},\ldots)\mid n = 1,2,\ldots\}\]
and
\[\liminf(a_n) = \sup\{\inf(a_n,a_{n+1},\ldots)\mid n = 1,2,\ldots\}.\]

Show that

$(a)$ $\liminf(a_n)\le \limsup(a_n)$.

$(b)$ $\limsup(a_n) = b$ iff for all $\epsilon > 0$, there is an $N$ so
that $b+\epsilon > a_n$ for all $n \ge N$ and for all $m$, there is an
$n \ge m$ so that $b-\epsilon < a_n$.

$(c)$ $a_n \to b$ iff $\limsup(a_n) = \liminf(a_n) = b$.

$(d)$ Let $a_n = (-1)^n$. Compute $\liminf(a_n)$, $\limsup(a_n)$.

\textbf{Proof.} For each $n$, put $b_n = \sup(a_n,a_{n+1},\ldots)$ and
$c_n = \inf(a_n,a_{n+1},\ldots)$.

$(a)$ We have $\limsup(a_n) = \inf(b_n)$ and $\liminf(a_n)= \sup(c_n)$.
Moreover, $b_n$ is an decreasing sequence, $c_n$ is an increasing
sequence, and $b_n \ge c_n$ for all $n$. Fix $n$, we claim that
$b_n \ge c_m$ for all $m$. Indeed, if $m \le n$, then
$c_m \le c_n \le b_n$. If $m > n$, then $c_m \le b_m \le b_n$. In any
case, we have $c_m \le b_n$.

By $b_n \ge c_m$ for all $m$, we get $g_n \ge \sup(c_m)$, which means
$b_n \ge \liminf(a_m)$. Note that this is true for all $n$. So
$\inf(b_n) \ge \liminf(a_m)$, which means
$\limsup(a_n) \ge \liminf(a_m)$.

$(b)$ For the $(\Rightarrow)$ side, suppose $\limsup(a_n) = b$, which
means $\inf(b_n) = b$. For $\epsilon > 0$, by definition of $\inf$,
there is $N$ such that $b \le b_N < b + \epsilon$. Because
$b_N=\sup(a_N,a_{N+1},\ldots)$, we have $a_n < b + \epsilon$ for all
$n\ge N$. On the other hand, for each $m$, by $b_N \ge b$, we have
$\sup(a_N, a_{N+1},\ldots) \ge b > b-\epsilon$. Hence by definition of
$\sup$, there exists $n\ge m$ such that $a_n > b-\epsilon$.

For the $(\Leftarrow)$ side, first, we fix $\epsilon > 0$. Because
$b + \epsilon > a_n$ for all $n \ge N$, we have
$b + \epsilon \ge \sup(a_N,a_{N+1},\ldots)$, hence $b+\epsilon \ge b_N$.
But $b_N \ge \inf(b_n)$, hence $b+\epsilon \ge \limsup(a_n)$. On the
other hand, for each $m$, because $b - \epsilon < a_n$ for some
$n \ge m$, we have $b-\epsilon < \sup(a_m,a_{m+1},\ldots)$. Hence
$b - \epsilon < b_m$ for all $m$, hence $b-\epsilon \le \inf(b_m)$. We
conclude that $b-\epsilon \le \limsup(a_n) \le b+ \epsilon$. This is
true for all $\epsilon>0$, so $b = \limsup(a_n)$.

$(c)$ Similar to $(b)$, we have $\liminf(a_n) = b$ iff for all
$\epsilon > 0$, there is an $N$ so that $b-\epsilon < a_n$ for all
$n\ge N$ and for all $m$, there is an $n \ge m$ so that
$b+\epsilon > a_n$.

Now, for the $(\Rightarrow)$ side, suppose $a_n \to b$, then for all
$\epsilon > 0$, there is an $N$ so that
$b + \epsilon > a_n > b-\epsilon$ for all $n\ge N$. Of course, for all
$m$, by choosing $n = \max\{N,m\}$, we have $n \ge m$ and
$b + \epsilon > a_n > b-\epsilon$. Hence, by $(b)$ and the above note,
$\limsup(a_n) = \liminf(a_n) = b$.

For the $(\Leftarrow)$ side, suppose $\limsup(a_n) = \liminf(a_n) = b$,
then for all $\epsilon > 0$, there are $N_1$ and $N_2$ so that
$b+\epsilon > a_n$ for all $n\ge N_1$ and $b-\epsilon < a_n$ for all
$n \ge N_2$. Let $N = \max\{N_1,N_2\}$, then for all $n \ge N$, we have
$b+\epsilon > a_n > b-\epsilon$. This implies that $a_n \to b$.

$(d)$ We have $b_n = 1$ for all $n$ and $c_n = -1$ for all $n$. So
$\limsup(a_n) = 1$ and $\liminf(a_n) = -1$. $\Box$

    \textbf{15.} Let $x_n$ be a sequence in $\mathbb{R}$ such that
$d(x_n,x_{n+1}) \le d(x_{n-1},x_n)/ 2$. Then show that $x_n$ is a Cauchy
sequence.

\textbf{Proof.} For each $n$, we have
$d(x_n, x_{n+1}) \le d(x_{n-1},x_n)/2 \le \ldots \le d(x_1,x_2)/2^{n-1}.$
So if $x_1 = x_2$, then $x_n$ will be a constant sequence, which is
always a Cauchy sequence. Now suppose that $d(x_1,x_2) > 0$. Let
$N \ge 1$. For $m > n >=N$, we have
\[ d(x_n, x_m) \leq d(x_n, x_{n+1}) + d(x_{n+1},x_{n+2}) + \ldots + d(x_{m-1}, x_m).\]

So
\[ d(x_n, x_m) \leq d(x_1,x_2)/2^{n-1} + d(x_1,x_2)/2^n + \ldots + d(x_1,x_2)/2^{m-2}.\]

Note that \[\begin{aligned}
\frac{1}{2^{n-1}} + \frac{1}{2^n} + \ldots + \frac{1}{2^{m-2}} &= \frac{1}{2^{m-2}}(2^{m-n-1}+2^{m-n-2}+\ldots+ 1)\\
&= \frac{1}{2^{m-2}}(2^{m-n}-1)\\
&= \frac{4}{2^n} - \frac{4}{2^m}.
\end{aligned}\]

By $4/2^n - 4/2^m < 4/2^n \le 4/2^N$, for $\epsilon > 0$, if we choose
$N$ such that $2^N > 4d(x_1,x_2)/\epsilon$, then we will have
$d(x_n, x_m) < \epsilon$ for all $m > n \ge N$. This means that $x_n$ is
a Cauchy sequence. $\Box$

    \textbf{28.} Let $x_n$ be a Cauchy sequence in $\mathbb{R}$ and let
$A_n = \sup\{x_n,x_{n+1},\ldots\}$ and
$B_n = \inf\{x_n,x_{n+1},\ldots\}$. Prove $A_n$ converges to the same
limit as $B_n$, which in turn is the same as the limit of $x_n$.

\textbf{Proof.} Because $x_n$ is a Cauchy sequence and $\mathbb{R}$ is
complete, we have $x_n$ converges to a point $x\in \mathbb{R}$. For
$\epsilon > 0$, let $N\in \mathbb{N}$ such that $|x_n-x| < \epsilon$ for
all $n \ge N$. For each $n\ge N$, we have $x-\epsilon < x_n$, hence
$A_n \ge x_n > x- \epsilon$. Also, for all $m \ge n$, we have
$x_m < x + \epsilon$. This implies that $x + \epsilon$ is an upper bound
of $A_n$, hence $A_n \le x+\epsilon$. We have show that
$|A_n - x| \leq \epsilon$ for every $n\ge N$, which means $A_n$
converges to $x$.

Similarly, we also have $B_n$ converges to $x$, so $A_n$ and $B_n$
converge to the same limit $x$ (which is the limit of $x_n$). $\Box$

    \textbf{29.} For any $x\in \mathbb{R}$, $x\geq 0$, use the axioms for
$\mathbb{R}$ to deduce the existence of $y\in \mathbb{R}$ such that
$y^2 = x$.

\textbf{Proof.} The case $x=0$ is trivial because we just take $y=0$.
Now suppose that $x > 0$. Let
$A = \{t\in \mathbb{R} \mid t \ge 0 \text{ and } t^2 \leq x\}$. Because
$0\in A$, $A$ is not empty. Let $N \in \mathbb{N}$ such that $n > x$. We
have $n \ge 1$, so $n^2 =n.n \ge n.1 > x$. We claim that $n$ is a upper
bound of $A$. Indeed, for $t\in A$, we have $t^2 \le x < N^2$, hence
$(n-t)(n+t) > 0$, hence $n - t > 0$. So $n > t$ for all $t \in A$.

Because $A$ is non-empty and is bounded from above by $n$, we can let
$a = \sup(A)$. It is clear that $a \geq 0$. Now, suppose $a^2 < x$ and
let $d = x - a^2$. We can choose $\epsilon > 0$ such that $\epsilon < 1$
and $\epsilon < \frac{d}{2a+1}$. Then because
$\epsilon(2a+\epsilon) < \epsilon(2a +1) < d = x-a^2$, we have
$a^2 + 2a \epsilon + \epsilon^2 < d$, which implies that
$a + \epsilon \in A$. Because $a$ is an upper bound of $A$, we have
$a + \epsilon \le a$, which is a contradiction. So $a^2 \geq x$.

If $a^2 > x$, similarly, we can find $\epsilon > 0$ such that
$\epsilon < a$ and $(a-\epsilon)^2 > x$. Put $c = a-\epsilon$, then
$0 < c < a$. For every $b \in A$, we have $b^2 \le x < c^2$, hence
$c^2 - b^2 > 0$, hence $(c-b)(c+b) > 0$, hence $c > b$. So $c$ is an
upper bound of $A$. Because $a$ is the least upper bound of $A$, we must
have $c \ge a$, which is a contradiction. So we must have $a^2 = x$.
$\Box$

    \textbf{30.} Use the axioms for $\mathbb{R}$ to prove the Archimedian
property: for every $x\in \mathbb{R}$ there exists an integer $N$ such
that $N > x$.

\textbf{Proof.} If $n \leq x$ for all $n = 1,2,\ldots$, then the
sequence $\{x_n\}$ with $x_n = n$ is increasing and bounded from above.
By the completeness axiom, $\{x_n\}$ must converge to a point
$a\in \mathbb{R}$. Pick $\epsilon = 1$, there exists $N\in \mathbb{N}$
such that $|x_n - a| < \epsilon$ for all $n \geq N$, . In particular, we
have $|N - a| < 1$ with $n= N$ and $|N+2 - a| < 1$ with $n=N+2$. The
first inequality implies $N < a + 1$ and the second inequality implies
$N > a + 1$. We get a contradiction. So there must exist an integer $N$
such that $N > x$. $\Box$

    \subsubsection*{Section 2.1}\label{section-2.1}
\addcontentsline{toc}{subsubsection}{Section 2.1}

    \textbf{1.} Show that $\mathbb{R}^2\backslash \{(0,0)\}$ is open in
$\mathbb{R}^2$.

\textbf{Proof.} Let $(x,y)\in \mathbb{R}^2\backslash \{(0,0)\}$. Because
$(x,y)\neq (0,0)$, we have $d((x,y),(0,0)) > 0$. Let
$\epsilon = d((x,y),(0,0))/2$, then because
$d((x,y), (0,0)) = 2\epsilon > \epsilon$, we have
$(0,0)\notin D((x,y),\epsilon)$. So
$D((x,y),\epsilon)) \subset \mathbb{R}^2\backslash \{(0,0)\}$. This
implies that $\mathbb{R}^2\backslash \{(0,0)\}$ is an open set. $\Box$

    \textbf{3.} Let $A \subset \mathbb{R}$ be open and
$B\subset \mathbb{R}^2$ be defined by
\[[ B = \{(x,y)\in\mathbb{R}^2 \mid x\in A\}.\]

Show that $B$ is open.

\textbf{Proof.} Let $(x,y)\in B$. Because $x\in A$, there exists
$\epsilon > 0$ such that the interval $(x-\epsilon, x+\epsilon)$ is a
subset of $A$. To show that $B$ is open, it is enough to show that
$D((x,y),\epsilon) \subset B$. Indeed, for $(a,b)\in D((x,y),\epsilon)$,
we have $d((x,y),(a,b)) < \epsilon$, hence
$\sqrt{(x-a)^2 +(y-b)^2} < \epsilon$, hence
\[|x-a|= \sqrt{(x-a)^2} \leq \sqrt{(x-a)^2 +(y-b)^2} < \epsilon.\]

So $a \in A$, which implies that $(a,b)\in B$. This is true for all
$(a,b) \in D((x,y),\epsilon)$. So $D((x,y),\epsilon) \subset B$. $\Box$


    % Add a bibliography block to the postdoc
    
    
    
    \end{document}
