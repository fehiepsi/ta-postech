
% Default to the notebook output style

    


% Inherit from the specified cell style.




    
\documentclass{article}

    
    
    \usepackage{graphicx} % Used to insert images
    \usepackage{adjustbox} % Used to constrain images to a maximum size 
    \usepackage{color} % Allow colors to be defined
    \usepackage{enumerate} % Needed for markdown enumerations to work
    \usepackage{geometry} % Used to adjust the document margins
    \usepackage{amsmath} % Equations
    \usepackage{amssymb} % Equations
    \usepackage[mathletters]{ucs} % Extended unicode (utf-8) support
    \usepackage[utf8x]{inputenc} % Allow utf-8 characters in the tex document
    \usepackage{fancyvrb} % verbatim replacement that allows latex
    \usepackage{grffile} % extends the file name processing of package graphics 
                         % to support a larger range 
    % The hyperref package gives us a pdf with properly built
    % internal navigation ('pdf bookmarks' for the table of contents,
    % internal cross-reference links, web links for URLs, etc.)
    \usepackage{hyperref}
    \usepackage{longtable} % longtable support required by pandoc >1.10
    

    

    
    % Prevent overflowing lines due to hard-to-break entities
    \sloppy 
    % Setup hyperref package
    \hypersetup{
      breaklinks=true,  % so long urls are correctly broken across lines
      colorlinks=true,
      urlcolor=blue,
      linkcolor=darkorange,
      citecolor=darkgreen,
      }
    % Slightly bigger margins than the latex defaults
    
    \geometry{verbose,tmargin=1in,bmargin=1in,lmargin=1in,rmargin=1in}
    
    

    \begin{document}
    
    
    
    
    

    
    \emph{Note.} In HW\#4, Exercise 4.c, the example is
$f(x,y) = xy^2/(x^2+y^4)$.

    \subsection*{Analysis I - homework - week
7}\label{analysis-i---homework---week-7}
\addcontentsline{toc}{subsection}{Analysis I - homework - week 7}

    \textbf{1.} $(a)$ Prove that if $A\subset \mathbb{R}^n$ is compact,
$B \subset\mathcal{C}(A, \mathbb{R}^m)$ is compact $\Leftrightarrow$ $B$
is closed, bounded, and equicontinuous.

$(b)$ Let $D = \{f \in\mathcal{C}([0,1], \mathbb{R}) \mid \|f\|\le 1\}$.
Show $D$ is closed and bounded, but is not compact.

\textbf{Sketch of proof.} $(a)$ For the $(\Rightarrow)$ side, each
sequence in $B$ has a subsequence converges uniformly to a function $f$.
Because $B$ is closed, $f\in B$. For the $(\Leftarrow)$ side, clearly
$B$ is closed and bounded. To show the equicontinuousness, use
$\epsilon/3$ trick. First, fix $x\in A$. For $\epsilon >0$,
$B \subset D(f_1,\epsilon/3)\cup \ldots \cup D(f_n,\epsilon/3)$ for some
$f_1,\ldots,f_n$ (look at Exercise 3 for a similar argument).

$(b)$ To show $D$ is not compact, let $f_n(x) = x^n$, then $\{f_n\}$
cannot have a subsequence converges (on the contrary, the limit of that
subsequence is continuous, but the limit must be $g$ with $g(x) = 1$ if
$x =1$ and $g(x)=0$ if $x < 1$). $\Box$

    \textbf{2.} Suppose $f:\mathbb{R}\to\mathbb{R}$ is continuous and
$f(1) = 7$. Suppose $f(x)$ is rational for all $x$. Prove $f$ is
constant.

\textbf{Sketch of proof.} If $f(x) < 7$ and $x < 1$, then $f([x,1])$ is
connected. So $[f(x) , 7] \subset f([x,1])$. $\Box$

    \textbf{3.} Let $\{f_n\}$ be a uniform convergent sequence of uniform
continuous functions. Prove that $\{f_n\}$ is equicontinuous.

\textbf{Sketch of proof.} In HW\#6, Exercise 2, we now that $f_n$
converges to a (uniformly) continuous function $f$. Fix a point
$a \in X$. Using $\epsilon/3$ trick again, for $\epsilon > 0$, there
exists $N$ such that $d(f_n(x),f(x)) < \epsilon/3$ for all $x\in X$, for
all $n > N$. Choose $\delta > 0$ such that $d(f(x),f(a)) < \epsilon /3$
if $d(x,a) < \delta$. Show that $d(f_n(x),f(a)) < \epsilon$ for all
$n > N$ and for all $x \in D(a,\delta)$. To conclude, note that
$\{f_1,f_2,...,f_N\}$ is a finite set, so this set is always
equicontinuous. Note that the assumption that $f_n$ is uniform
continuous is not necessary, continuous is enough. $\Box$

    \textbf{4.} Let $f:\mathbb{R}\to\mathbb{R}$ be a uniform limit of
polynomials. Prove that $f$ is a polynomial.

\textbf{Sketch of proof.} Let $\{p_n\}$ be a sequence of polynomials
which converges to $f$. First show that if $(p_n - p_m)$ is a polynomial
and bounded, then it is constant. So there exists $N$ such that
$a_n = p_n - p_N$ is constant for all $n > N$. By the uniform
convergence of $\{p_n\}$, $a_n$ converges to some point
$a\in \mathbb{R}$. Show that $f = a + p_N$.

    \textbf{5.} Prove that if

$(a)$ $f_n$, $g(x)$ continuous, $0 \le x < \infty$,

$(b)$ $|f_n| < g(x)$, $n = 1,2,\ldots$, $0 \le x < \infty$,

$(c)$ $f_n(x) \to f(x)$ uniformly, $0 \le x \le R$, for any
$R < \infty$,

$(d)$ $\int_0^{\infty}g(x)\,dx < \infty$,

then
\[\lim_{n\to\infty}\int_0^{\infty}f_n(x)\,dx= \int_0^{\infty}f(x)\,dx.\]

    \textbf{Sketch of proof.} Using $\epsilon/3$ trick, let $\epsilon > 0$.
There is $R>0$ such that $\int_R^{\infty} g(x)\,dx < \epsilon/3$. We
have $f_n(x) \to f(x)$ uniformly on $[0,R]$, so there exists $N$ such
that $\int_0^R|f_n(x) - f(x)| < \epsilon/3$ if $n > N$. Note that
$|f(x)| \le g(x)$ for all $x\in [0,\infty)$, so
$\int_R^{\infty} f_n(x)\,dx < \epsilon/3$ and
$\int_R^{\infty} f(x)\,dx < \epsilon/3$.

    \textbf{6.} $(a)$ \emph{Young's inequality:} Let $p>1$ with $1/p+1/q=1$.
For $a,b,t>0$, prove that \[ab\le \frac{a^pt^p}{p}+\frac{b^qt^{-q}}{q}\]
and that $ab$ is the minimum value of the right side.

$(b)$ \emph{Holder's inequality:} Let $a_k,b_k\ge 0$ and $p > 1$, and
$1/p + 1/q = 1$. Prove that
\[\sum_1^n a_kb_k\le \left(\sum_1^n a_k^p\right)^{1/p} \left(\sum_1^n b_k^q\right)^{1/q}.\]

$(c)$ \emph{Minkowski's inequality:} Let $a_k,b_k\ge 0$ and $p > 1$.
Prove that
\[\left(\sum_1^n (a_k+b_k)^p\right)^{1/p}\le \left(\sum_1^n a_k^p\right)^{1/p} + \left(\sum_1^n b_k^p\right)^{1/p}.\]

\textbf{Sketch of proof.} $(a)$ Put $c=at$, $d=bt^{-1}$, we want to
prove that \[ cd \le \frac{c^p}{p} + \frac{d^q}{q}.\]

To prove, let logarithm of two sides and note that $\ln(x)$ is a convex
function:
$\ln(\lambda x + (1-\lambda)y)\le \lambda\ln(x)+(1-\lambda)\ln(y)$ (the
second derivative of $\ln(x)$ on $(0,\infty)$ is negative). The equal
happens when $c^p = d^q$.

$(b)$ Excluding the trivial cases, put
$c_k = a_k/\left(\sum_1^n a_k^p\right)^{1/p}$,
$d_k = b_k/\left(\sum_1^n b_k^q\right)^{1/q}$. Then
$\sum c_k^p = \sum d_k^q = 1$. Apply $(a)$.

$(c)$ We have
\[\sum(a_k+b_k)^p = \sum(a_k+b_k)^{p-1}a_k + \sum(a_k+b_k)^{p-1}b_k,\]
and
\[ \sum(a_k+b_k)^{p-1}a_k \le \left(\sum (a_k+b_k)^{(p-1)q}\right)^{1/q}(\sum a_k^p)^{1/p},\]

\[\sum(a_k+b_k)^{p-1}b_k \le \left(\sum (a_k+b_k)^{(p-1)q}\right)^{1/q}(\sum b_k^p)^{1/p}.\]

    \textbf{7.} Let $f:X\to X$ be a continuous function, where $X$ is a
complete metric space (such as $\mathbb{R}$) satisfying
$d(f(x), f(y)) < d(x, y)$ for all $x, y\in X$. Must $f$ have a fixed
point? Discuss. What if $X$ is compact?

\textbf{Sketch of proof.} Consider $f: (0,\infty)\to (0,\infty)$ with
$x\mapsto x/2$. If $X$ is compact, consider the function
$\varphi : X\to\mathbb{R}$ with $x\mapsto d(f(x),x)$. Show that
$\varphi(X)$ is compact and has the infimum $0$. Conclude that there
exists $x$ such that $\varphi(x) = 0$.

    \textbf{8.} $(a)$ Define
$I : \mathcal{C}([0,1], \mathbb{R}) \to \mathbb{R}$ as follow:

\[I(f) =\int_0^1 f(x)\,dx.\]

Prove that I is continuous.

$(b)$ Show that

\[\left\{ f\in\mathcal{C}([0,1], \mathbb{R}) \mid \int_0^1 f(x) \, dx \in (0,3)\right\} \]
is open.

\textbf{Sketch of proof.} $(a)$ $|(f) - I(g)| \le d(f,g)$.

$(b)$ Consequence of $(a)$.

    \textbf{9.} Let $f_n:[a,b]\to \mathbb{R}$ be uniformly bounded
continuous functions. Set
\[F_n(x) = \int_0^x f_n(t)\,dt,\qquad a\le x \le b.\]

Prove $F_n$ has a uniformly convergent subsequence.

\textbf{Sketch of proof.} First, using the uniformly boundedness of
$f_n$ to show that $\{F_n\}$ is bounded and equicontinuous. Then apply
Arzela-Ascoli's theorem.

    \textbf{10.} Let
$T : \mathcal{C}_b([0, r], \mathbb{R}) \to \mathcal{C}_b([0, r], \mathbb{R})$
be defined by \[ T(f)(x) = \alpha f(x) +\int_0^x k(x,y)f(y)\,dy,\] where
$\alpha$ is constant and $k : [0, r]^2 \to\mathbb{R}$ is a continuous
function. Prove that if
\[ |\alpha|+\sup_{x\in [ 0, r]}\int_0^x |k(x,y)|\,dy = \lambda < 1,\]
then $T$ is a contraction.

\textbf{Sketch of proof.} We have, for each $x$,
\[\| T(f)(x)-T(g)(x)\| \le |\alpha| \|f-g\| + \int_0^x|k(x,y)|\|f-g\|\,dy \le \lambda \|f-g\|.\]


    % Add a bibliography block to the postdoc
    
    
    
    \end{document}
